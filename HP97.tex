
\documentclass[a4paper,12pt]{article}
%%%%%%%%%%%%%%%%%%%%%%%%%%%%%%%%%%%%%%%%%%%%%%%%%%%%%%%%%%%%%%%%%%%%%%%%%%%%%%%%%%%%%%%%%%%%%%%%%%%%%%%%%%%%%%%%%%%%%%%%%%%%%%%%%%%%%%%%%%%%%%%%%%%%%%%%%%%%%%%%%%%%%%%%%%%%%%%%%%%%%%%%%%%%%%%%%%%%%%%%%%%%%%%%%%%%%%%%%%%%%%%%%%%%%%%%%%%%%%%%%%%%%%%%%%%%
\usepackage{graphicx,hyperref,mathpple,amsmath,exscale,setspace,xcolor}
\usepackage[left=20mm,right=20mm,top=20mm,bottom=20mm]{geometry}
\usepackage{pdflscape,showkeys,changepage}
\usepackage[round]{natbib}

\setcounter{MaxMatrixCols}{10}
%TCIDATA{OutputFilter=LATEX.DLL}
%TCIDATA{Version=5.50.0.2953}
%TCIDATA{<META NAME="SaveForMode" CONTENT="2">}
%TCIDATA{BibliographyScheme=BibTeX}
%TCIDATA{Created=Wednesday, May 03, 2023 13:45:06}
%TCIDATA{LastRevised=Friday, October 27, 2023 17:18:21}
%TCIDATA{<META NAME="GraphicsSave" CONTENT="32">}
%TCIDATA{<META NAME="DocumentShell" CONTENT="Standard LaTeX\Blank - Standard LaTeX Article">}
%TCIDATA{CSTFile=40 LaTeX article.cst}

\let\oldref\ref
\AtBeginDocument{
\let\oldref\ref\renewcommand{\ref}[1]{(\oldref{#1})}
\newcommand{\bsq}{\begin{subequations}}\newcommand{\esq}{\end{subequations}}
\newcommand{\bls}{\begin{landscape}}\newcommand{\els}{\end{landscape}}
\renewcommand\showkeyslabelformat[1]{{\parbox[t]{\marginparwidth}{\raggedright\footnotesize\url{#1}}}}
\newcommand{\intxt}[1]{\intertext{#1}}\newcommand{\BAW}[1]{\begin{adjustwidth}{-#1mm}{-5mm}}\newcommand{\EAW}{\end{adjustwidth}}
\newcommand{\vsp}[1]{\vspace*{#1mm}}\newcommand{\hsp}[1]{\hspace*{#1mm}}  }
\makeatletter
\renewcommand*{\@fnsymbol}[1]{\ensuremath{\ifcase#1\or *\or
    \#\or \star\or \bowtie\or \star\star\or \ddagger\ddagger \else\@ctrerr\fi}}
\makeatother
\allowdisplaybreaks
\IfFileExists{C:/swp55/TCITeX/TeX/LaTeX/SWmacros/tcilatex.tex}{\input{tcilatex}}{}
\graphicspath{{../graphics/}{graphics/}}
\newcommand{\dble}{1.77}
\newcommand{\sngl}{1.23}
\definecolor{myred}{rgb}{.50,.10,.10}
\definecolor{mygrn}{rgb}{.10,.35,.10}
\definecolor{myblu}{rgb}{.10,.10,.35}
\hypersetup{colorlinks,citecolor=myblu,filecolor=mygrn,linkcolor=myred,urlcolor=mygrn,breaklinks=true}
\setstretch{\sngl}

\begin{document}


\section{HP97}

The Filter of Hodrick and Prescott\ (1997, HP-Filter) can be expressed as an
SSM following a UC\ model structure as:\bsq\label{HP0}%
\begin{align}
y_{t}& =y_{t}^{\ast }+y_{t}^{c}  \label{HP0a} \\
\Delta ^{2}y_{t}^{\ast }& =\varepsilon _{1t}  \label{HP0b} \\
y_{t}^{c}& =\phi \varepsilon _{2t},  \label{HP0c}
\end{align}%
\esq where $y_{t}$ is (generally 100 times) the log of GDP, and $\varepsilon
_{1t}$ and $\varepsilon _{2t}$ are $N(0,1)$. The standard deviation $\phi $
is the (square root of the) smoothing parameter, generally set to $40$,
implying a value of `$\lambda $' of $1600$.

The number shock to named shock mapping is:%
\begin{equation}
\begin{bmatrix}
\varepsilon _{1t} \\ 
\varepsilon _{2t}%
\end{bmatrix}%
=%
\begin{bmatrix}
\varepsilon _{t}^{\ast } \\ 
\varepsilon _{t}^{c}%
\end{bmatrix}%
.
\end{equation}

\section{Shock recovery SSM}

\subsection{SSM with lagged states}

Kurz's (2018) SSM has the following general from:\bsq\label{SSM}%
\begin{align}
\mathsf{Measurement}& :\quad Z_{t}=D_{1}X_{t}+D_{2}X_{t-1}+R\varepsilon _{t}
\label{ssm1} \\
\mathsf{State}& :\quad X_{t}=AX_{t-1}+C\varepsilon _{t},  \label{ssm2}
\end{align}%
\esq where $\varepsilon _{t}\sim MN(0,I_{m})$, $D_{1},D_{2},A,R$ are $C$ are
conformable system matrices, $Z_{t}$ the observed variable and $X_{t}$ the
latent state variable.

\subsection{HP97 SSM\ for shock recovery}

To assess recovery, re-write the model in \ref{HP0} in `\emph{shock recovery}%
' State Space Form (SSF). That is, collect all observables in $Z_{t}$ and
all shocks (and other state variables)\ in $X_{t}$ to yield:%
\begin{align}
\Delta ^{2}y_{t}& =\Delta ^{2}y_{t}^{\ast }+\Delta ^{2}y_{t}^{c}  \notag \\
& =\varepsilon _{1t}+\phi \Delta ^{2}\varepsilon _{2t}  \notag \\
& =\varepsilon _{1t}+\phi \varepsilon _{2t}-2\phi \varepsilon _{2t-1}+\phi
\varepsilon _{2t-2},  \label{Z}
\end{align}%
where $\Delta ^{2}y_{t}=Z_{t}$ is the only observed variable.

Note: The estimates of the shocks from the Kalman Filter $E_{t}X_{t}=E_{t}%
\begin{bmatrix}
\varepsilon _{t} & \varepsilon _{2t} & \varepsilon _{2t-1}%
\end{bmatrix}%
^{\prime }$ will be linked by the identity:%
\begin{equation*}
E_{t}\varepsilon _{2t}=\phi E_{t}\varepsilon _{1t}
\end{equation*}%
and from the Kalman Smoother $E_{T}X_{t}=E_{T}%
\begin{bmatrix}
\varepsilon _{t} & \varepsilon _{2t} & \varepsilon _{2t-1}%
\end{bmatrix}%
^{\prime }$ by the identity: 
\begin{equation}
\Delta ^{2}E_{T}\varepsilon _{1t}=\frac{1}{\phi }E_{T}\varepsilon _{2t-2}.
\label{KS}
\end{equation}%
With $\varepsilon _{1t}=\Delta ^{2}y_{t}^{\ast }$ and $\varepsilon _{2t}=%
\frac{1}{\phi }y_{t}^{c}$, this means that the output from the standard
HP--Filter will give the identity:%
\begin{align*}
\Delta ^{4}y_{t}^{\ast }& =\frac{1}{\phi }y_{t-2}^{c} \\
\Delta ^{4}\text{HP--trend}_{t}& =\frac{1}{\phi ^{2}}\text{HP--cycle}_{t-2}.
\end{align*}%
Indeed, running a regression of $\Delta ^{4}$HP--trend$_{t}$ on HP--cycle$%
_{t-2}$ (without an intercept) yields indeed a regression coefficient of $%
0.000625=1/1600$ when applied to US--GDP data that was HP--Filtered with the
smoothing parameter set to $\lambda =1600=40^{2}$. The regression fit is
perfect, with an $R^{2}$ of $1$ and a residual sum of squares of $0.$

The Measurement and State equations corresponding to \ref{Z} are:\bsq\label%
{K0SSM}%
\begin{align}
\mathsf{Measurement}:\quad Z_{t}& =D_{1}X_{t}+D_{2}X_{t-1}+R\varepsilon _{t}
\notag \\
& =\varepsilon _{1t}+\phi \varepsilon _{2t}-2\phi \varepsilon _{2t-1}+\phi
\varepsilon _{2t-2} \\[2mm]
Z_{t}& =\underbrace{%
\begin{bmatrix}
1 & \phi & 0%
\end{bmatrix}%
}_{D_{1}}\underbrace{%
\begin{bmatrix}
\varepsilon _{1t} \\ 
\varepsilon _{2t} \\ 
\varepsilon _{2t-1}%
\end{bmatrix}%
}_{X_{t}}+\underbrace{%
\begin{bmatrix}
0 & -2\phi & \phi%
\end{bmatrix}%
}_{D_{2}}\underbrace{%
\begin{bmatrix}
\varepsilon _{1t-1} \\ 
\varepsilon _{2t-1} \\ 
\varepsilon _{2t-2}%
\end{bmatrix}%
}_{X_{t-1}}+\underbrace{%
\begin{bmatrix}
0 & 0%
\end{bmatrix}%
}_{R}\underbrace{%
\begin{bmatrix}
\varepsilon _{1t} \\ 
\varepsilon _{2t}%
\end{bmatrix}%
} \\
\mathsf{State}:\quad X_{t}& =AX_{t-1}+C\varepsilon _{t},  \notag \\[5mm]
\underbrace{%
\begin{bmatrix}
\varepsilon _{1t} \\ 
\varepsilon _{2t} \\ 
\varepsilon _{2t-1}%
\end{bmatrix}%
}_{X_{t}}& =\underbrace{%
\begin{bmatrix}
0 & 0 & 0 \\ 
0 & 0 & 0 \\ 
0 & 1 & 0%
\end{bmatrix}%
}_{A}\underbrace{%
\begin{bmatrix}
\varepsilon _{1t-1} \\ 
\varepsilon _{2t-1} \\ 
\varepsilon _{2t-2}%
\end{bmatrix}%
}_{X_{t-1}}+\underbrace{%
\begin{bmatrix}
1 & 0 \\ 
0 & 1 \\ 
0 & 0%
\end{bmatrix}%
}_{C}\underbrace{%
\begin{bmatrix}
\varepsilon _{1t} \\ 
\varepsilon _{2t}%
\end{bmatrix}%
}_{\varepsilon _{t}}.
\end{align}%
\esq

\end{document}
