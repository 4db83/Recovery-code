
\documentclass[a4paper,final,12pt]{article}
%%%%%%%%%%%%%%%%%%%%%%%%%%%%%%%%%%%%%%%%%%%%%%%%%%%%%%%%%%%%%%%%%%%%%%%%%%%%%%%%%%%%%%%%%%%%%%%%%%%%%%%%%%%%%%%%%%%%%%%%%%%%%%%%%%%%%%%%%%%%%%%%%%%%%%%%%%%%%%%%%%%%%%%%%%%%%%%%%%%%%%%%%%%%%%%%%%%%%%%%%%%%%%%%%%%%%%%%%%%%%%%%%%%%%%%%%%%%%%%%%%%%%%%%%%%%
\usepackage{graphicx,hyperref,mathpple,amsmath,exscale,setspace,xcolor}
\usepackage[left=20mm,right=20mm,top=20mm,bottom=20mm]{geometry}
\usepackage{pdflscape,showkeys,changepage}
\usepackage[TABBOTCAP]{subfigure}
\usepackage[round]{natbib}

\setcounter{MaxMatrixCols}{10}
%TCIDATA{OutputFilter=LATEX.DLL}
%TCIDATA{Version=5.50.0.2953}
%TCIDATA{<META NAME="SaveForMode" CONTENT="2">}
%TCIDATA{BibliographyScheme=BibTeX}
%TCIDATA{Created=Wednesday, May 03, 2023 13:45:06}
%TCIDATA{LastRevised=Friday, January 12, 2024 16:30:04}
%TCIDATA{<META NAME="GraphicsSave" CONTENT="32">}
%TCIDATA{<META NAME="DocumentShell" CONTENT="Standard LaTeX\Blank - Standard LaTeX Article">}
%TCIDATA{CSTFile=40 LaTeX article.cst}

\AtBeginDocument{
\let\oldref\ref\renewcommand{\ref}[1]{(\oldref{#1})}
\newcommand{\bsq}{\begin{subequations}}\newcommand{\esq}{\end{subequations}}
\newcommand{\bls}{\begin{landscape}}\newcommand{\els}{\end{landscape}}
\renewcommand\showkeyslabelformat[1]{{\parbox[t]{\marginparwidth}{\raggedright\footnotesize\url{#1}}}}
\newcommand{\intxt}[1]{\intertext{#1}}\newcommand{\BAW}[1]{\begin{adjustwidth}{-#1mm}{-5mm}}\newcommand{\EAW}{\end{adjustwidth}}
\newcommand{\vsp}[1]{\vspace*{#1mm}}\newcommand{\hsp}[1]{\hspace*{#1mm}}  }
\renewcommand\section{\@startsection{section}{1}{\z@}{-2.5ex \@plus -1ex \@minus -.2ex}{0.01ex \@plus.2ex}{\Large\bfseries}}
\renewcommand\subsection{\@startsection{subsection}{1}{\z@}{-1.5ex \@plus -1ex \@minus -.2ex}{0.01ex \@plus.2ex}{\large\bfseries}}
\makeatletter
\renewcommand*{\@fnsymbol}[1]{\ensuremath{\ifcase#1\or *\or
    \#\or \star\or \bowtie\or \star\star\or \ddagger\ddagger \else\@ctrerr\fi}}
\makeatother
\allowdisplaybreaks
\IfFileExists{C:/swp55/TCITeX/TeX/LaTeX/SWmacros/tcilatex.tex}{\input{tcilatex}}{}
\graphicspath{{./graphics/}{../graphics/}{../../graphics/}}
\definecolor{myred}{rgb}{.50,.10,.10}
\definecolor{mygrn}{rgb}{.10,.35,.10}
\definecolor{myblu}{rgb}{.10,.10,.35}
\hypersetup{colorlinks,citecolor=myblu,filecolor=mygrn,linkcolor=myred,urlcolor=mygrn,breaklinks=true}
\setstretch{1.2345}
\setlength{\parskip}{7pt}

\begin{document}


\section{Clark87}

The Clark (1987) Unobserved Component (UC) model is a generalisation of the
HP--Filter (a local linear trend model) that can be expressed in State Space
Form (SSF) as:\bsq\label{clark0}%
\begin{align}
y_{t}& =y_{t}^{\ast }+\tilde{y}_{t} \\
\Delta y_{t}^{\ast }& =g_{t-1}+\sigma _{1}\varepsilon _{1t} \\
\Delta g_{t}& =\sigma _{2}\varepsilon _{2t} \\
a(L)\tilde{y}_{t}& =\sigma _{3}\varepsilon _{3t},
\end{align}%
\esq where $y_{t}$ is (100 times) the log of GDP, and the shocks $\left\{
\varepsilon _{it}\right\} _{i=1}^{3}$ are mutually uncorrelated $i.i.d.$ $%
N(0,1)$, with standard deviations $\left\{ \sigma _{i}\right\} _{i=1}^{3}$,
and $a(L)$ is a lag polynomial commonly assumed to be a stable AR(2), ie., $%
a(L)=(1-a_{1}L-a_{2}L^{2})$, with the roots of $a(L)$ being outside the unit
circle. The cycle $\tilde{y}_{t}$ is allowed to be serially correlated.
There are 3 shocks in the model.

The `\emph{numbered} \emph{shock}' to `\emph{named shock}' mapping is:%
\begin{equation}
\begin{bmatrix}
\varepsilon _{1t} \\ 
\varepsilon _{2t} \\ 
\varepsilon _{3t}%
\end{bmatrix}%
=%
\begin{bmatrix}
\varepsilon _{t}^{y^{\ast }} \\ 
\varepsilon _{t}^{g} \\ 
\varepsilon _{t}^{\tilde{y}}%
\end{bmatrix}%
,
\end{equation}%
where $\varepsilon _{t}^{y^{\ast }}$, $\varepsilon _{t}^{g}$, and $%
\varepsilon _{t}^{\tilde{y}}$ are the trend (permanent), trend growth and
cycle shocks, respectively.

\section{Shock recovery}

\subsection{State Space Models with lagged states}

Kurz's (2018) SSM has the following general from:\bsq\label{SSM}%
\begin{align}
\mathsf{Measurement}& :\quad Z_{t}=D_{1}X_{t}+D_{2}X_{t-1}+R\varepsilon _{t}
\label{ssm1} \\
\mathsf{State}& :\quad X_{t}=AX_{t-1}+C\varepsilon _{t},  \label{ssm2}
\end{align}%
\esq where $\varepsilon _{t}\sim MN(0_{m},I_{m})$, $D_{1},D_{2},A,R$ are $C$
are conformable system matrices, $Z_{t}$ the observed variable and $X_{t}$
the latent state variable, and $m$ is the number of shocks $\left\{
\varepsilon _{it}\right\} _{i=1}^{m}$.

\subsection{Clark87 in `\emph{shock recovery}' SSF}

To assess shock recovery, write the model in \ref{clark0} in `\emph{shock
recovery}' SSF by collecting all observable variables in $Z_{t}$ and all
shocks (and other latent state variables)\ in $X_{t}$. Differencing $y_{t}$
and $y_{t}^{c}$ twice, and re-arranging the relations in \ref{clark0}\ then
yields:%
\begin{align}
\Delta ^{2}y_{t}& =\Delta ^{2}y_{t}^{\ast }+\Delta ^{2}\tilde{y}_{t}  \notag
\\
& =\sigma _{1}\Delta \varepsilon _{1t}+\Delta g_{t-1}+\Delta ^{2}\tilde{y}%
_{t}  \notag \\
& =\sigma _{1}\Delta \varepsilon _{1t}+\sigma _{2}\varepsilon
_{2t-1}+a(L)^{-1}\sigma _{3}\Delta ^{2}\varepsilon _{3t}  \notag \\
\Leftrightarrow a(L)\Delta ^{2}y_{t}& =\sigma _{1}a(L)\Delta \varepsilon
_{1t}+\sigma _{2}a(L)\varepsilon _{2t-1}+\sigma _{3}\Delta ^{2}\varepsilon
_{3t}  \label{clark1}
\end{align}%
where $a(L)\Delta ^{2}y_{t}$ is the only observed variable. Re-writing \ref%
{clark1} in more convenient form for the SSF yields:%
\begin{equation}
\begin{split}
\underbrace{a(L)\Delta ^{2}y_{t}}_{Z_{t}}& =a(L)\sigma _{1}\Delta
\varepsilon _{1t}+a(L)\sigma _{2}\varepsilon _{2t-1}+\sigma _{3}\Delta
^{2}\varepsilon _{3t} \\[-5.5mm]
& =\sigma _{1}\Delta \varepsilon _{1t}-a_{1}\sigma _{1}\Delta \varepsilon
_{1t-1}-a_{2}\sigma _{1}\Delta \varepsilon _{1t-2}+\sigma _{2}\varepsilon
_{2t-1}-a_{1}\sigma _{2}\varepsilon _{2t-2} \\
& -\ a_{2}\sigma _{2}\varepsilon _{2t-3}+\sigma _{3}\Delta \varepsilon
_{3t}-\sigma _{3}\Delta \varepsilon _{3t-1}.
\end{split}
\label{Z}
\end{equation}

The Measurement and State equations of the `\emph{shock recovery}' SSF
corresponding to the relations in \ref{Z} are then given by:\bsq\label{K0SSM}%
\begin{eqnarray}
\mathsf{Measurement}:\quad Z_{t} &=&D_{1}X_{t}+D_{2}X_{t-1}+R\varepsilon _{t}
\\[-17mm]
Z_{t} &=&\underbrace{%
\begin{bmatrix}
0 & 0 & 0 & \sigma _{1} & -a_{1}\sigma _{1} & 0 & 0 & \sigma _{3}%
\end{bmatrix}%
}_{D_{1}}\underbrace{%
\begin{bmatrix}
\varepsilon _{1t} \\ 
\varepsilon _{2t} \\ 
\varepsilon _{3t} \\ 
\Delta \varepsilon _{1t} \\ 
\Delta \varepsilon _{1t-1} \\ 
\varepsilon _{2t-1} \\ 
\varepsilon _{2t-2} \\ 
\Delta \varepsilon _{3t}%
\end{bmatrix}%
}_{X_{t}} \\[-17mm]
&+&\underbrace{%
\begin{bmatrix}
0 & \sigma _{2} & 0 & 0 & -a_{2}\sigma _{1} & -a_{1}\sigma _{2} & 
-a_{2}\sigma _{2} & -\sigma _{3}%
\end{bmatrix}%
}_{D_{2}}\underbrace{%
\begin{bmatrix}
\varepsilon _{1t-1} \\ 
\varepsilon _{2t-1} \\ 
\varepsilon _{3t-1} \\ 
\Delta \varepsilon _{1t-1} \\ 
\Delta \varepsilon _{1t-2} \\ 
\varepsilon _{2t-2} \\ 
\varepsilon _{2t-3} \\ 
\Delta \varepsilon _{3t-1}%
\end{bmatrix}%
}_{X_{t-1}}+\underbrace{%
\begin{bmatrix}
0 & 0 & 0%
\end{bmatrix}%
}_{R}\underbrace{%
\begin{bmatrix}
\varepsilon _{1t} \\ 
\varepsilon _{2t} \\ 
\varepsilon _{3t}%
\end{bmatrix}%
}_{\varepsilon _{t}}  \notag
\end{eqnarray}

\vspace*{-12mm}

\begin{align}
\mathsf{State}:\quad X_{t}& =AX_{t-1}+C\varepsilon _{t} \\
\underbrace{%
\begin{bmatrix}
\varepsilon _{1t} \\ 
\varepsilon _{2t} \\ 
\varepsilon _{3t} \\ 
\Delta \varepsilon _{1t} \\ 
\Delta \varepsilon _{1t-1} \\ 
\varepsilon _{2t-1} \\ 
\varepsilon _{2t-2} \\ 
\Delta \varepsilon _{3t}%
\end{bmatrix}%
}_{X_{t}}& =\underbrace{%
\begin{bmatrix}
0 & 0 & 0 & 0 & 0 & 0 & 0 & 0 \\ 
0 & 0 & 0 & 0 & 0 & 0 & 0 & 0 \\ 
0 & 0 & 0 & 0 & 0 & 0 & 0 & 0 \\ 
-1 & 0 & 0 & 0 & 0 & 0 & 0 & 0 \\ 
0 & 0 & 0 & 1 & 0 & 0 & 0 & 0 \\ 
0 & 1 & 0 & 0 & 0 & 0 & 0 & 0 \\ 
0 & 0 & 0 & 0 & 0 & 1 & 0 & 0 \\ 
0 & 0 & -1 & 0 & 0 & 0 & 0 & 0%
\end{bmatrix}%
}_{A}\underbrace{%
\begin{bmatrix}
\varepsilon _{1t-1} \\ 
\varepsilon _{2t-1} \\ 
\varepsilon _{3t-1} \\ 
\Delta \varepsilon _{1t-1} \\ 
\Delta \varepsilon _{1t-2} \\ 
\varepsilon _{2t-2} \\ 
\varepsilon _{2t-3} \\ 
\Delta \varepsilon _{3t-1}%
\end{bmatrix}%
}_{X_{t-1}}+\underbrace{%
\begin{bmatrix}
1 & 0 & 0 \\ 
0 & 1 & 0 \\ 
0 & 0 & 1 \\ 
1 & 0 & 0 \\ 
0 & 0 & 0 \\ 
0 & 0 & 0 \\ 
0 & 0 & 0 \\ 
0 & 0 & 1%
\end{bmatrix}%
}_{C}\underbrace{%
\begin{bmatrix}
\varepsilon _{1t} \\ 
\varepsilon _{2t} \\ 
\varepsilon _{3t}%
\end{bmatrix}%
}_{\varepsilon _{t}}.
\end{align}%
\esq

\subsection{Shock recovery}

The diagonal of the steady-state variance/covariance matrix of the smoothed
and filtered states $X_{t}$ denoted by $P_{t|T}^{\ast }$ and $P_{t|t}^{\ast
} $, respectively, are:%
\begin{equation}
\begin{tabular}{ccccc}
\hline
Shocks &  & $P_{t|T}^{\ast }$ &  & $P_{t|t}^{\ast }$ \\ \hline
$\varepsilon _{1t}$ &  & $0.5469$ &  & $0.5989$ \\ 
$\varepsilon _{2t}$ &  & $0.9870$ &  & $1.0000$ \\ 
$\varepsilon _{3t}$ &  & $0.4661$ &  & $0.5153$ \\ \hline
\end{tabular}%
~,  \label{Pstar}
\end{equation}%
indicating that the trend growth shock $\varepsilon _{2t}=\varepsilon
_{t}^{g}$ cannot be recovered ($P^{\ast }\approx 1$), while the cycle shock $%
\varepsilon _{3t}=\varepsilon _{t}^{\tilde{y}}$ and the trend shock $%
\varepsilon _{1t}=\varepsilon _{t}^{y^{\ast }}$ have $P^{\ast }\approx 0.5$,
suggesting that there are recovery difficulties. Note that $P_{t|t}^{\ast
}=1 $ implies that Kalman filtered estimates of $\varepsilon _{2t}$ are 
\emph{exactly} zero for all $t$. This can be seen from the entries on the
left hand side of \ref{KFS} which shows filtered estimates of shocks,
(smoothed estimates are shown on the right).%
\begin{equation}
\begin{tabular}{rcrcrrr}
\hline
\multicolumn{3}{c}{Filtered} & ~~ & \multicolumn{3}{c}{Smoothed$\ $} \\ 
$E_{t}\varepsilon _{1t}\ \ $ & $E_{t}\varepsilon _{2t}$ & $E_{t}\varepsilon
_{3t}\ \ $ & ~~ & $E_{T}\varepsilon _{1t}\ \ $ & $E_{T}\varepsilon _{2t}\ \ $
& $E_{T}\varepsilon _{3t}\ \ $ \\ \hline
$-0.7088$ & $0$ & $-0.7792$ &  & $0.3098$ & $-0.0160$ & $0.0344$ \\ 
$-0.8039$ & $0$ & $-0.8837$ &  & $-0.5269$ & $0.0043$ & $-0.5442$ \\ 
$-0.2554$ & $0$ & $-0.2808$ &  & $-0.1336$ & $0.0094$ & $0.1118$ \\ 
$0.0179$ & $0$ & $0.0196$ &  & $0.6757$ & $-0.0166$ & $0.4942$ \\ 
$-0.5488$ & $0$ & $-0.6032$ &  & $-0.2535$ & $-0.0068$ & $-0.4492$ \\ 
$-0.3244$ & $0$ & $-0.3566$ &  & $-0.3713$ & $0.0075$ & $-0.2660$ \\ 
$0.1304$ & $0$ & $0.1434$ &  & $0.2703$ & $-0.0029$ & $0.3115$ \\ 
$-0.2821$ & $0$ & $-0.3101$ &  & $-0.6689$ & $0.0228$ & $0.0549$ \\ \hline
\end{tabular}%
~.  \label{KFS}
\end{equation}%
In \autoref{fig:1}, simulated states and corresponding Kalman smoothed
estimates are plotted for the shocks of interest.

\begin{figure}[h!]
\centering
\includegraphics[width=1\textwidth,trim={0 0 0
0},clip,angle=00]{Clark87_plots_KS.pdf}
\caption{Comparison of true shocks and Kalman Smoothed estimates $\protect%
\varepsilon _{t|T}$.}
\label{fig:1}
\end{figure}
The correlation between the true (simulated) and estimated Kalman smoothed
shocks can be analyzed by simply computing $\mathrm{Corr}(X_{t},\hat{X}%
_{t|T})$, where $X_{t}=%
\begin{bmatrix}
\varepsilon _{1t} & \varepsilon _{2t} & \varepsilon _{3t}%
\end{bmatrix}%
^{\prime }$ and $\hat{X}_{t|T}=E_{T}X_{t}=E_{T}%
\begin{bmatrix}
\varepsilon _{1t} & \varepsilon _{2t} & \varepsilon _{2t-1}%
\end{bmatrix}%
^{\prime }$, which yields:%
\begin{equation}
\begin{tabular}{lcc}
\hline
Shocks &  & $\mathrm{Corr}(X_{t},\hat{X}_{t|T})$ \\ \hline
$\varepsilon _{1t}$ &  & $0.6736$ \\ 
$\varepsilon _{2t}$ &  & $0.1184$ \\ 
$\varepsilon _{3t}$ &  & $0.7304$ \\ \hline
\end{tabular}%
~.  \label{Corr}
\end{equation}%
From \ref{Corr} one can see that the estimated $\varepsilon _{1t}$ and $%
\varepsilon _{3t}$ shocks are rather weakly correlated with the true values, 
$0.6736$ and $0.7304$, respectively, while the estimated $\varepsilon _{2t}$
shock is nearly uncorrelated ($0.1184$).\ These facts can also be seen from %
\autoref{fig:1} below.

\subsection{Shock Identities}

As was the case for the HP-Filter, the Kalman filtered estimates of the
trend and cycle shocks $\varepsilon _{1t}$ and $\varepsilon _{3t}$ are
linked by the (filter) identity:%
\begin{equation}
E_{t}\varepsilon _{1t}=0.909694E_{t}\varepsilon _{3t}.  \label{KF}
\end{equation}%
Kalman smoothed estimates give the following \emph{dynamic} identities:%
\begin{equation}
\Delta ^{2}E_{T}\varepsilon _{2t}=-0.038486\Delta E_{T}\varepsilon _{1t},
\label{KS1}
\end{equation}%
and also:%
\begin{align}
\Delta ^{2}E_{T}\varepsilon _{3t}& =-1.935746\Delta E_{T}\varepsilon
_{1t-1}+1.659427E_{T}\varepsilon _{3t-1}-1.760937E_{T}\varepsilon _{3t-2}
\label{KS2a} \\
\Delta ^{2}E_{T}\varepsilon _{3t}& =50.297162\Delta E_{T}\varepsilon
_{2t-1}+1.659427E_{T}\varepsilon _{3t-1}-1.760937E_{T}\varepsilon _{3t-2}.
\label{KS2b}
\end{align}

The contemporaneous correlations of the shocks from the Kalman filtered and
smoothed estimates are, respectively: 
\begin{equation*}
\begin{tabular}{lccr}
\multicolumn{4}{c}{$\mathrm{Corr}(\hat{X}_{t|t},\hat{X}_{t|t})$} \\ \hline
Shocks & {$\varepsilon _{1t}$} & {$\varepsilon _{2t}$} & $\varepsilon _{3t}$
\\ \hline
\multicolumn{1}{c}{$\varepsilon _{1t}$} & \multicolumn{1}{r}{$1.0000$} & $%
\mathrm{NaN}$ & $1.0000$ \\ 
\multicolumn{1}{c}{$\varepsilon _{2t}$} & $\mathrm{NaN}$ & $\mathrm{NaN}$ & 
\multicolumn{1}{c}{$\mathrm{NaN}$} \\ 
\multicolumn{1}{c}{$\varepsilon _{3t}$} & \multicolumn{1}{r}{$1.0000$} & $%
\mathrm{NaN}$ & $1.0000$ \\ \hline
\end{tabular}%
\text{ \ \ \ \ and \ \ \ \ }%
\begin{tabular}{lccc}
\multicolumn{4}{c}{$\mathrm{Corr}(\hat{X}_{t|T},\hat{X}_{t|T})$} \\ \hline
Shocks & {$\varepsilon _{1t}$} & {$\varepsilon _{2t}$} & $\varepsilon _{3t}$
\\ \hline
\multicolumn{1}{c}{$\varepsilon _{1t}$} & \multicolumn{1}{r}{$1.0000$} & 
\multicolumn{1}{r}{$-0.1104$} & \multicolumn{1}{r}{$1.0000$} \\ 
\multicolumn{1}{c}{$\varepsilon _{2t}$} & $-0.1104$ & \multicolumn{1}{r}{$%
1.0000$} & $-0.1446$ \\ 
\multicolumn{1}{c}{$\varepsilon _{3t}$} & \multicolumn{1}{r}{$0.8403$} & 
\multicolumn{1}{r}{$-0.1446$} & \multicolumn{1}{r}{$1.0000$} \\ \hline
\end{tabular}%
~\text{.}
\end{equation*}%
Note that the $\left\{ {\varepsilon _{it}}\right\} _{i=1}^{3}${\ }were
generated as mutually uncorrelated $i.i.d.$ $N(0,1)$ processes, while their `%
\emph{large sample}' estimates are correlated. The code\ $\mathtt{Clark87.m}$
replicates the output summarized here. \ 

\subsection{Maximum Likelihood estimation}

The (standard)\ SSF\ for ML\ estimation for the model in \ref{clark0} is:%
\begin{align}
y_{t}& =%
\begin{bmatrix}
1 & 0 & 1 & 0%
\end{bmatrix}%
\begin{bmatrix}
y_{t}^{\ast } \\ 
g_{t} \\ 
\tilde{y}_{t} \\ 
\tilde{y}_{t-1}%
\end{bmatrix}%
+0\varepsilon _{t} \\[4mm]
\begin{bmatrix}
y_{t}^{\ast } \\ 
g_{t} \\ 
\tilde{y}_{t} \\ 
\tilde{y}_{t-1}%
\end{bmatrix}%
& =%
\begin{bmatrix}
1 & 1 & 0 & 0 \\ 
0 & 1 & 0 & 0 \\ 
0 & 0 & a_{1} & a_{2} \\ 
0 & 0 & 1 & 0%
\end{bmatrix}%
\begin{bmatrix}
y_{t-1}^{\ast } \\ 
g_{t-1} \\ 
\tilde{y}_{t-1} \\ 
\tilde{y}_{t-2}%
\end{bmatrix}%
+%
\begin{bmatrix}
\sigma _{1} & 0 & 0 \\ 
0 & \sigma _{2} & 0 \\ 
0 & 0 & \sigma _{3} \\ 
0 & 0 & 0%
\end{bmatrix}%
\begin{bmatrix}
\varepsilon _{1t} \\ 
\varepsilon _{2t} \\ 
\varepsilon _{3t}%
\end{bmatrix}%
.
\end{align}%
Below are estimates and plots of Clark's 87 model fitted to U.S. GDP\ data
from $1947$:Q2 to $2019$:Q4. 
\begin{table}[h!]
\centering
\includegraphics[width=.75\textwidth,trim={0 0 0 25},clip]{Clark_MLE.pdf} \vspace*{-2.5mm}
\caption{Clark (1987) model MLE parameter estimates for the U.S. from $1947$:Q2 to $2019$:Q4.}
\label{tab:MLE}
\end{table}

\begin{figure}[p!]
\centering
\includegraphics[angle=00, width=1\textwidth,trim={0 0 0 0},clip]{Clark_SSM_Smoothed.pdf} \vspace*{-2.5mm}
\caption{Smoother estimates of scaled shocks $\eta_{it} = \sigma_i\varepsilon_{it}, \forall i=1,2,3$.}
\label{fig:KS}
\end{figure}

\begin{figure}[p!]
\centering
\includegraphics[angle=00, width=1\textwidth,trim={0 0 0 0},clip]{Clark_SSM} \vspace*{-2.5mm}
\caption{Filtered and Smoothed estimates trend, trend growth and cycle.}
\label{fig:KFS}
\end{figure}

\end{document}
