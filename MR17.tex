
\documentclass[a4paper,12pt]{article}
%%%%%%%%%%%%%%%%%%%%%%%%%%%%%%%%%%%%%%%%%%%%%%%%%%%%%%%%%%%%%%%%%%%%%%%%%%%%%%%%%%%%%%%%%%%%%%%%%%%%%%%%%%%%%%%%%%%%%%%%%%%%%%%%%%%%%%%%%%%%%%%%%%%%%%%%%%%%%%%%%%%%%%%%%%%%%%%%%%%%%%%%%%%%%%%%%%%%%%%%%%%%%%%%%%%%%%%%%%%%%%%%%%%%%%%%%%%%%%%%%%%%%%%%%%%%
\usepackage{graphicx,hyperref,mathpple,amsmath,exscale,setspace,xcolor}
\usepackage[left=20mm,right=20mm,top=20mm,bottom=20mm]{geometry}
\usepackage{pdflscape,showkeys,changepage}
\usepackage[round]{natbib}

\setcounter{MaxMatrixCols}{14}
%TCIDATA{OutputFilter=LATEX.DLL}
%TCIDATA{Version=5.50.0.2953}
%TCIDATA{<META NAME="SaveForMode" CONTENT="2">}
%TCIDATA{BibliographyScheme=BibTeX}
%TCIDATA{Created=Wednesday, May 03, 2023 13:45:06}
%TCIDATA{LastRevised=Wednesday, October 25, 2023 17:24:05}
%TCIDATA{<META NAME="GraphicsSave" CONTENT="32">}
%TCIDATA{<META NAME="DocumentShell" CONTENT="Standard LaTeX\Blank - Standard LaTeX Article">}
%TCIDATA{CSTFile=40 LaTeX article.cst}

\let\oldref\ref
\AtBeginDocument{
\let\oldref\ref\renewcommand{\ref}[1]{(\oldref{#1})}
\newcommand{\bsq}{\begin{subequations}}\newcommand{\esq}{\end{subequations}}
\newcommand{\bls}{\begin{landscape}}\newcommand{\els}{\end{landscape}}
\renewcommand\showkeyslabelformat[1]{{\parbox[t]{\marginparwidth}{\raggedright\footnotesize\url{#1}}}}
\newcommand{\intxt}[1]{\intertext{#1}}\newcommand{\BAW}[1]{\begin{adjustwidth}{-#1mm}{-5mm}}\newcommand{\EAW}{\end{adjustwidth}}
\newcommand{\vsp}[1]{\vspace*{#1mm}}\newcommand{\hsp}[1]{\hspace*{#1mm}}  }
\makeatletter
\renewcommand*{\@fnsymbol}[1]{\ensuremath{\ifcase#1\or *\or
    \#\or \star\or \bowtie\or \star\star\or \ddagger\ddagger \else\@ctrerr\fi}}
\makeatother
\allowdisplaybreaks
\IfFileExists{C:/swp55/TCITeX/TeX/LaTeX/SWmacros/tcilatex.tex}{\input{tcilatex}}{}
\newcommand{\dble}{1.77}
\newcommand{\sngl}{1.23}
\definecolor{myred}{rgb}{.50,.10,.10}
\definecolor{mygrn}{rgb}{.10,.35,.10}
\definecolor{myblu}{rgb}{.10,.10,.35}
\hypersetup{colorlinks,citecolor=myblu,filecolor=mygrn,linkcolor=myred,urlcolor=mygrn,breaklinks=true}
\setstretch{\sngl}

\begin{document}


\section{MR17}

McCririck and Rees (2017, MR17) is effectively an extension of LW03's model,
adding an equation for Okun's law. To keep these two models comparable, the
notation and number labelling of shocks is kept as in LW03. Following the
description of the model on page 16 in MR17 (Appendix A: Estimating the
Model), the model takes the form:%
\begin{align}
\tilde{y}_{t}& =a_{y,1}\tilde{y}_{t-1}+a_{y,2}\tilde{y}_{t-2}-\frac{a_{r}}{2}%
\sum_{i=1}^{2}(r_{t-i}-r_{t-i}^{\ast })+\sigma _{1}\varepsilon _{1t}
\label{MR1} \\
\pi _{t}& =(1-\beta _{1})\pi _{t}^{e}+\frac{\beta _{1}}{3}\sum_{i=1}^{3}\pi
_{t-i}+\beta _{2}(u_{t-1}-u_{t-1}^{\ast })+\sigma _{2}\varepsilon _{2t}
\label{MR2} \\
\Delta z_{t}& =\sigma _{3}\varepsilon _{3t},  \label{MR3} \\
\Delta y_{t}^{\ast }& =g_{t}+\sigma _{4}\varepsilon _{4t}\qquad (\text{MR17
use }g_{t}\text{ in paper, we use }g_{t-1}\text{ as in LW03 in SSM below})
\label{MR4} \\
\Delta g_{t}& =\sigma _{5}\varepsilon _{5t}  \label{MR5} \\
\Delta u_{t}^{\ast }& =\sigma _{6}\varepsilon _{6t}  \label{MR6} \\
u_{t}& =u_{t}^{\ast }+\beta (.4\tilde{y}_{t}+.3\tilde{y}_{t-1}+.2\tilde{y}%
_{t-2}+.1\tilde{y}_{t-3})+\sigma _{7}\varepsilon _{7t}  \label{MR7}
\end{align}%
where $\tilde{y}_{t}=(y_{t}-y_{t}^{\ast })$ and $r_{t}^{\ast }=4g_{t}+z_{t}$%
, and the numbered shock to named shock mapping is: 
\begin{equation}
\begin{bmatrix}
\varepsilon _{1t} \\ 
\varepsilon _{2t} \\ 
\varepsilon _{3t} \\ 
\varepsilon _{4t} \\ 
\varepsilon _{5t} \\ 
\varepsilon _{6t} \\ 
\varepsilon _{7t}%
\end{bmatrix}%
=%
\begin{bmatrix}
\varepsilon _{t}^{\tilde{y}} \\ 
\varepsilon _{t}^{\pi } \\ 
\varepsilon _{t}^{z} \\ 
\varepsilon _{t}^{y^{\ast }} \\ 
\varepsilon _{t}^{g} \\ 
\varepsilon _{t}^{u^{\ast }} \\ 
\varepsilon _{t}^{u}%
\end{bmatrix}%
.
\end{equation}

\section{Shock recovery SSM}

\subsection{SSM with lagged states}

Kurz's (2018) SSM has the following general from:\bsq\label{SSM}%
\begin{align}
\mathsf{Measurement}& :\quad Z_{t}=D_{1}X_{t}+D_{2}X_{t-1}+R\varepsilon _{t}
\label{ssm1} \\
\mathsf{State}& :\quad X_{t}=AX_{t-1}+C\varepsilon _{t},  \label{ssm2}
\end{align}%
\esq where $\varepsilon _{t}\sim MN(0,I_{m})$, $D_{1},D_{2},A,R$ are $C$ are
conformable system matrices, $Z_{t}$ the observed variable and $X_{t}$ the
latent state variable.

\subsection{MR17 SSM for shock recovery}

To assess recovery, re-write the model in `\emph{shock recovery}' form. That
is, collect all observables in $Z_{t}$, and all shocks (and other state
variables)\ in state vector $X_{t}$.\bsq\label{ssm0}%
\begin{align}
\mathsf{Measurement}& :\;\quad Z_{1t}=y_{t}^{\ast }-a_{y,1}y_{t-1}^{\ast
}-a_{y,2}y_{t-2}^{\ast }-\frac{a_{r}}{2}\left( r_{t-1}^{\ast }+r_{t-2}^{\ast
}\right) +\sigma _{1}\varepsilon _{1t} \\
\phantom{\mathsf{Measurement}}& \,\,\;\;\phantom{:\quad}Z_{2t}=-\beta
_{2}u_{t-1}^{\ast }+\sigma _{2}\varepsilon _{2t} \\
\phantom{\mathsf{Measurement}}& \,\,\;\;\phantom{:\quad}Z_{3t}=u_{t}^{\ast
}-\beta (.4y_{t}^{\ast }+.3y_{t-1}^{\ast }+.2y_{t-2}^{\ast }+.1y_{t-3}^{\ast
})+\sigma _{7}\varepsilon _{7t} \\
\mathsf{State}& :\quad \Delta y_{t}^{\ast }=g_{t-1}+\sigma _{4}\varepsilon
_{4t} \\
\phantom{\mathsf{State}}& \;\,\;\phantom{:\quad}\Delta g_{t}=\sigma
_{5}\varepsilon _{5t} \\
\phantom{\mathsf{State}}& \;\,\;\phantom{:\quad}\Delta u_{t}^{\ast }=\sigma
_{6}\varepsilon _{6t},  \label{drstar2} \\
\phantom{\mathsf{State}}& \;\,\;\phantom{:\quad}\Delta r_{t}^{\ast }=4\sigma
_{5}\varepsilon _{5t}+\sigma _{3}\varepsilon _{3t},
\end{align}%
\esq with the observables $Z_{t}$ in the measurement equations defined as:%
\vsp{-3}\bsq\label{ssmO} 
\begin{align}
Z_{1t}& =y_{t}-\left( \sum_{i=1}^{2}(a_{y,i}y_{t-i})-\frac{a_{r}}{2}%
\sum_{i=1}^{2}r_{t-i}\right) \\
Z_{2t}& =\pi _{t}-\left( (1-\beta _{1})\pi _{t}^{e}+\frac{\beta _{1}}{3}%
\sum_{i=1}^{3}\pi _{t-i}+\beta _{2}u_{t-1}\right) \\
Z_{3t}& =u_{t}-\beta (.4y_{t}+.3y_{t-1}+.2y_{t-2}+.1y_{t-3}).
\end{align}%
\esq

The `\emph{shock recovery}' SSF\ corresponding to \ref{ssm0} is then: \bsq%
\label{K0SSM}\BAW{9}

\begin{align}
\mathsf{Measurement}& :\quad Z_{t}=D_{1}X_{t}+D_{2}X_{t-1}+R\varepsilon _{t}
\notag \\[-4mm]
\underbrace{\left[ 
\begin{array}{c}
Z_{1t} \\ 
Z_{2t} \\ 
Z_{3t}%
\end{array}%
\right] }_{Z_{t}}& =\underbrace{%
\begin{bmatrix}
1 & -a_{y,1} & -a_{y,2} & 0 & 0 & 0 & 0 & \sigma _{1} & 0 & 0 & 0 & 0 & 0 & 0
\\ 
0 & 0 & 0 & 0 & 0 & 0 & 0 & 0 & \sigma _{2} & 0 & 0 & 0 & 0 & 0 \\ 
-.4\beta & -.3\beta & -.2\beta & 0 & 0 & 0 & 1 & 0 & 0 & 0 & 0 & 0 & 0 & 
\sigma _{7}%
\end{bmatrix}%
}_{D_{1}}\underbrace{\left[ 
\begin{array}{c}
y_{t}^{\ast } \\ 
y_{t-1}^{\ast } \\ 
y_{t-2}^{\ast } \\ 
g_{t} \\ 
r_{t}^{\ast } \\ 
r_{t-1}^{\ast } \\ 
u_{t}^{\ast } \\ 
\varepsilon _{1t} \\ 
\varepsilon _{2t} \\ 
\varepsilon _{3t} \\ 
\varepsilon _{4t} \\ 
\varepsilon _{5t} \\ 
\varepsilon _{6t} \\ 
\varepsilon _{7t}%
\end{array}%
\right] }_{X_{t}} \\
& +\underbrace{%
\begin{bmatrix}
0 & 0 & 0 & 0 & -\frac{a_{r}}{2} & -\frac{a_{r}}{2} & 0 & 0 & 0 & 0 & 0 & 0
& 0 & 0 \\ 
0 & 0 & 0 & 0 & 0 & 0 & -\beta _{2} & 0 & 0 & 0 & 0 & 0 & 0 & 0 \\ 
0 & 0 & -.1\beta & 0 & 0 & 0 & 0 & 0 & 0 & 0 & 0 & 0 & 0 & 0%
\end{bmatrix}%
}_{D_{2}}\underbrace{\left[ 
\begin{array}{c}
y_{t-1}^{\ast } \\ 
y_{t-2}^{\ast } \\ 
y_{t-3}^{\ast } \\ 
g_{t-1} \\ 
r_{t-1}^{\ast } \\ 
r_{t-2}^{\ast } \\ 
u_{t-1}^{\ast } \\ 
\varepsilon _{1t-1} \\ 
\varepsilon _{2t-1} \\ 
\varepsilon _{3t-1} \\ 
\varepsilon _{4t-1} \\ 
\varepsilon _{5t-1} \\ 
\varepsilon _{6t-1} \\ 
\varepsilon _{7t-1}%
\end{array}%
\right] }_{X_{t-1}}+\underbrace{\boldsymbol{0}_{2\times 5}}_{R}\underbrace{%
\begin{bmatrix}
\varepsilon _{1t} \\ 
\varepsilon _{2t} \\ 
\varepsilon _{3t} \\ 
\varepsilon _{4t} \\ 
\varepsilon _{5t} \\ 
\varepsilon _{6t} \\ 
\varepsilon _{7t}%
\end{bmatrix}%
}_{\varepsilon _{t}} \\
\mathsf{State}:\quad X_{t}& =AX_{t-1}+C\varepsilon _{t},  \notag \\[5mm]
\underbrace{\left[ 
\begin{array}{c}
y_{t}^{\ast } \\ 
y_{t-1}^{\ast } \\ 
y_{t-2}^{\ast } \\ 
g_{t} \\ 
r_{t}^{\ast } \\ 
r_{t-1}^{\ast } \\ 
u_{t}^{\ast } \\ 
\varepsilon _{1t} \\ 
\varepsilon _{2t} \\ 
\varepsilon _{3t} \\ 
\varepsilon _{4t} \\ 
\varepsilon _{5t} \\ 
\varepsilon _{6t} \\ 
\varepsilon _{7t}%
\end{array}%
\right] }_{X_{t}}& =\underbrace{%
\begin{bmatrix}
1 & 0 & 0 & 1 & 0 & 0 & 0 & 0 & 0 & 0 & 0 & 0 & 0 & 0 \\ 
1 & 0 & 0 & 0 & 0 & 0 & 0 & 0 & 0 & 0 & 0 & 0 & 0 & 0 \\ 
0 & 1 & 0 & 0 & 0 & 0 & 0 & 0 & 0 & 0 & 0 & 0 & 0 & 0 \\ 
0 & 0 & 0 & 1 & 0 & 0 & 0 & 0 & 0 & 0 & 0 & 0 & 0 & 0 \\ 
0 & 0 & 0 & 0 & 1 & 0 & 0 & 0 & 0 & 0 & 0 & 0 & 0 & 0 \\ 
0 & 0 & 0 & 0 & 1 & 0 & 0 & 0 & 0 & 0 & 0 & 0 & 0 & 0 \\ 
0 & 0 & 0 & 0 & 0 & 0 & 1 & 0 & 0 & 0 & 0 & 0 & 0 & 0 \\ 
0 & 0 & 0 & 0 & 0 & 0 & 0 & 0 & 0 & 0 & 0 & 0 & 0 & 0 \\ 
0 & 0 & 0 & 0 & 0 & 0 & 0 & 0 & 0 & 0 & 0 & 0 & 0 & 0 \\ 
0 & 0 & 0 & 0 & 0 & 0 & 0 & 0 & 0 & 0 & 0 & 0 & 0 & 0 \\ 
0 & 0 & 0 & 0 & 0 & 0 & 0 & 0 & 0 & 0 & 0 & 0 & 0 & 0 \\ 
0 & 0 & 0 & 0 & 0 & 0 & 0 & 0 & 0 & 0 & 0 & 0 & 0 & 0 \\ 
0 & 0 & 0 & 0 & 0 & 0 & 0 & 0 & 0 & 0 & 0 & 0 & 0 & 0 \\ 
0 & 0 & 0 & 0 & 0 & 0 & 0 & 0 & 0 & 0 & 0 & 0 & 0 & 0%
\end{bmatrix}%
}_{A}\underbrace{\left[ 
\begin{array}{c}
y_{t-1}^{\ast } \\ 
y_{t-2}^{\ast } \\ 
y_{t-3}^{\ast } \\ 
g_{t-1} \\ 
r_{t-1}^{\ast } \\ 
r_{t-2}^{\ast } \\ 
u_{t-1}^{\ast } \\ 
\varepsilon _{1t-1} \\ 
\varepsilon _{2t-1} \\ 
\varepsilon _{3t-1} \\ 
\varepsilon _{4t-1} \\ 
\varepsilon _{5t-1} \\ 
\varepsilon _{6t-1} \\ 
\varepsilon _{7t-1}%
\end{array}%
\right] }_{X_{t-1}}+\underbrace{%
\begin{bmatrix}
0 & 0 & 0 & \sigma _{4} & 0 & 0 & 0 \\ 
0 & 0 & 0 & 0 & 0 & 0 & 0 \\ 
0 & 0 & 0 & 0 & 0 & 0 & 0 \\ 
0 & 0 & 0 & 0 & \sigma _{5} & 0 & 0 \\ 
0 & 0 & \sigma _{3} & 0 & 4\sigma _{5} & 0 & 0 \\ 
0 & 0 & 0 & 0 & 0 & 0 & 0 \\ 
0 & 0 & 0 & 0 & 0 & \sigma _{6} & 0 \\ 
1 & 0 & 0 & 0 & 0 & 0 & 0 \\ 
0 & 1 & 0 & 0 & 0 & 0 & 0 \\ 
0 & 0 & 1 & 0 & 0 & 0 & 0 \\ 
0 & 0 & 0 & 1 & 0 & 0 & 0 \\ 
0 & 0 & 0 & 0 & 1 & 0 & 0 \\ 
0 & 0 & 0 & 0 & 0 & 1 & 0 \\ 
0 & 0 & 0 & 0 & 0 & 0 & 1%
\end{bmatrix}%
}_{C}\underbrace{%
\begin{bmatrix}
\varepsilon _{1t} \\ 
\varepsilon _{2t} \\ 
\varepsilon _{3t} \\ 
\varepsilon _{4t} \\ 
\varepsilon _{5t} \\ 
\varepsilon _{6t} \\ 
\varepsilon _{7t}%
\end{bmatrix}%
}_{\varepsilon _{t}}
\end{align}%
\EAW\esq

\subsubsection{Correlation between true change in natural rate and estimate}

The correlation between the true and estimated $\Delta r_{t}^{\ast }$ from
the SSM can be constructed from the relation:%
\begin{equation}
\rho =0.5\frac{\mathrm{Var}(\Delta r_{t}^{\ast })+\mathrm{Var}(E_{T}\Delta
r_{t}^{\ast })-\phi }{\sigma (\Delta r_{t}^{\ast })\sigma (E_{T}\Delta
r_{t}^{\ast })},
\end{equation}
where $\mathrm{Var}(\Delta r_{t}^{\ast })=4^{2}c^{2}\sigma _{5}^{2}+\sigma
_{3}^{2}$, $\sigma (\Delta r_{t}^{\ast })=\sqrt{\mathrm{Var}(\Delta
r_{t}^{\ast })}$, and $\mathrm{Var}(E_{T}\Delta r_{t}^{\ast })$ can be
computed from simulating from the true model, applying the Kalman Filter and
Smoother to get $E_{T}\Delta r_{t}^{\ast }$ and then computing the sample
variance of $E_{T}\Delta r_{t}^{\ast }$ as an estimate of $\mathrm{Var}%
(E_{T}\Delta r_{t}^{\ast })$.

To obtain $\phi $, add $\Delta r_{t}^{\ast }$ to the state-vector $X_{t}$
and augment the remaining matrices to be conformable. The required $\phi $
term is then the entry of $\mathrm{diag}(P_{t|T}^{\ast })$ that corresponds
to $\Delta r_{t}^{\ast }$, which will be the very last element\ (see also
LW03.pdf how this is done).

\bigskip

\bigskip

\bigskip

\bigskip

\bigskip

\bigskip

\end{document}
