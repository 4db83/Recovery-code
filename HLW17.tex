
\documentclass[a4paper,12pt]{article}
%%%%%%%%%%%%%%%%%%%%%%%%%%%%%%%%%%%%%%%%%%%%%%%%%%%%%%%%%%%%%%%%%%%%%%%%%%%%%%%%%%%%%%%%%%%%%%%%%%%%%%%%%%%%%%%%%%%%%%%%%%%%%%%%%%%%%%%%%%%%%%%%%%%%%%%%%%%%%%%%%%%%%%%%%%%%%%%%%%%%%%%%%%%%%%%%%%%%%%%%%%%%%%%%%%%%%%%%%%%%%%%%%%%%%%%%%%%%%%%%%%%%%%%%%%%%
\usepackage{graphicx,hyperref,mathpple,amsmath,exscale,setspace,xcolor}
\usepackage[left=20mm,right=20mm,top=20mm,bottom=20mm]{geometry}
\usepackage{pdflscape,showkeys,changepage}
\usepackage[round]{natbib}

\setcounter{MaxMatrixCols}{11}
%TCIDATA{OutputFilter=LATEX.DLL}
%TCIDATA{Version=5.50.0.2953}
%TCIDATA{<META NAME="SaveForMode" CONTENT="2">}
%TCIDATA{BibliographyScheme=BibTeX}
%TCIDATA{Created=Wednesday, May 03, 2023 13:45:06}
%TCIDATA{LastRevised=Monday, October 23, 2023 14:47:13}
%TCIDATA{<META NAME="GraphicsSave" CONTENT="32">}
%TCIDATA{<META NAME="DocumentShell" CONTENT="Standard LaTeX\Blank - Standard LaTeX Article">}
%TCIDATA{CSTFile=40 LaTeX article.cst}

\let\oldref\ref
\AtBeginDocument{
\let\oldref\ref\renewcommand{\ref}[1]{(\oldref{#1})}
\newcommand{\bsq}{\begin{subequations}}\newcommand{\esq}{\end{subequations}}
\newcommand{\bls}{\begin{landscape}}\newcommand{\els}{\end{landscape}}
\renewcommand\showkeyslabelformat[1]{{\parbox[t]{\marginparwidth}{\raggedright\footnotesize\url{#1}}}}
\newcommand{\intxt}[1]{\intertext{#1}}\newcommand{\BAW}[1]{\begin{adjustwidth}{-#1mm}{-5mm}}\newcommand{\EAW}{\end{adjustwidth}}
\newcommand{\vsp}[1]{\vspace*{#1mm}}\newcommand{\hsp}[1]{\hspace*{#1mm}}  }
\makeatletter
\renewcommand*{\@fnsymbol}[1]{\ensuremath{\ifcase#1\or *\or
    \#\or \star\or \bowtie\or \star\star\or \ddagger\ddagger \else\@ctrerr\fi}}
\makeatother
\allowdisplaybreaks
\IfFileExists{C:/swp55/TCITeX/TeX/LaTeX/SWmacros/tcilatex.tex}{\input{tcilatex}}{}
\newcommand{\dble}{1.77}
\newcommand{\sngl}{1.23}
\definecolor{myred}{rgb}{.50,.10,.10}
\definecolor{mygrn}{rgb}{.10,.35,.10}
\definecolor{myblu}{rgb}{.10,.10,.35}
\hypersetup{colorlinks,citecolor=myblu,filecolor=mygrn,linkcolor=myred,urlcolor=mygrn,breaklinks=true}
\setstretch{\sngl}

\begin{document}


\section{HLW's (2017) State-Space Model (SSM) Form}

HLW17 use the same standard State-Space Form (SSF) as in LW03, taking the
form:% (see their \texttt{%
%LW\_Code\_Guide.pdf} file \texttt{LW\_replication.zip} from the NYFED
%website):%
\begin{align*}
\mathsf{Measurement}& :\quad \mathbf{y}_{t}=\mathbf{Ax}_{t}+\mathbf{H}%
\boldsymbol{\xi }_{t}+\boldsymbol{R}^{1/2}\boldsymbol{\varepsilon }_{t}^{%
\mathbf{y}} \\
\mathsf{State}& :\quad \boldsymbol{\xi }_{t}=\mathbf{F}\boldsymbol{\xi }%
_{t-1}+\boldsymbol{Q}^{1/2}\boldsymbol{\varepsilon }_{t}^{\boldsymbol{\xi }}
\end{align*}%
where (note the different inflation specification)%
\begin{align*}
\mathbf{y}_{t}& =%
\begin{bmatrix}
y_{t} & \pi _{t}%
\end{bmatrix}%
^{\prime }, \\
\mathbf{x}_{t}& =%
\begin{bmatrix}
y_{t-1} & y_{t-2} & r_{t-1} & r_{t-2} & \pi _{t-1} & \pi _{t-2,4}%
\end{bmatrix}%
^{\prime }, \\
\boldsymbol{\xi }_{t}& =%
\begin{bmatrix}
y_{t}^{\ast } & y_{t-1}^{\ast } & y_{t-2}^{\ast } & g_{t-1} & g_{t-2} &
z_{t-1} & z_{t-2}%
\end{bmatrix}%
^{\prime }, \\
\mathbf{A}& =%
\begin{bmatrix}
a_{y,1} & a_{y,2} & \frac{a_{r}}{2} & \frac{a_{r}}{2} & 0 & 0 \\
b_{y} & 0 & 0 & 0 & b_{\pi } & 1-b_{\pi }%
\end{bmatrix}%
, \\
\mathbf{H}& =%
\begin{bmatrix}
1 & -a_{y,1} & -a_{y,2} & -4\frac{a_{r}}{2} & -4\frac{a_{r}}{2} & -\frac{%
a_{r}}{2} & -\frac{a_{r}}{2} \\
0 & -b_{y} & 0 & 0 & 0 & 0 & 0%
\end{bmatrix}%
.
\end{align*}

\noindent \textbf{Note:}\ This\ follows the notation used in the
documentation file:\ `\texttt{HLW\_Code\_Guide.pdf}' included in the zip
file: `\texttt{HLW\_replication.zip}' that contains the replication code of
HLW17 and which is available from the NYFED\ website at: %
\url{https://www.newyorkfed.org/medialibrary/media/research/economists/williams/data/HLW_Code.zip}%
.

Compared to LW03, the inflation dynamics are now changed and do not include
oil price inflation and core import inflation anymore. Also, the constant $c$
is now held fixed at $1$, rather than estimated due to `\emph{identification
problems}', so that $r_{t}^{\ast }=4g_{t}+z_{t}$. They write to this on page
S63: `\emph{In Laubach and Williams (2003), we estimated this relationship
and found a coefficient of close to unity. Because this relationship is not
well identified in the data, we chose to impose a coefficient of unity.}'. In
the description of their SSM, the constant $c$ has thus been removed for
comparability, but in the SSM used for recovery in Section 2, it is left in
the equations to make the adaption of the code easiest to the
HLW23-post-COVID19 version.

In the construction of $r_{t}^{\ast }$, trend growth $g_{t}$ is again
annualized, but not in the state equations for $g_{t}$. That is, in the
matrices above, the entries in $\mathbf{H(}1,4:5\mathbf{)}$ corresponding to
trend growth are multiplied by 4 (as well as $c$) in the code that performs
the estimation (see \texttt{unpack.parameters.stage3.R} in \texttt{%
HLW\_replication.zip} files, line 22, which reads):\

\texttt{H[1, 4:5] \TEXTsymbol{<}- -parameters[3] * 2 \# -a\_r/2 (annualized)}%
.

\noindent The standard deviations of the shocks are denoted by $%
\begin{bmatrix}
\sigma _{\tilde{y}} & \sigma _{\pi } & \sigma _{y^{\ast }} & \sigma _{g} &
\sigma _{z}%
\end{bmatrix}%
^{\prime }$ in the documentation in `\texttt{HLW\_Code\_Guide.pdf}'.

\pagebreak

\subsection{SSM of HLW17}

The measurement equation is:\bsq\label{obs0}
\begin{align}
\mathbf{y}_{t}& =\mathbf{Ax}_{t}+\mathbf{H}\xi _{t}+\boldsymbol{R}^{1/2}%
\boldsymbol{\varepsilon }_{t}^{\mathbf{y}}  \notag \\
\underbrace{%
\begin{bmatrix}
y_{t} \\
\pi _{t}%
\end{bmatrix}%
}_{\mathbf{y}_{t}}& =\underbrace{%
\begin{bmatrix}
a_{y,1} & a_{y,2} & \frac{a_{r}}{2} & \frac{a_{r}}{2} & 0 & 0 \\
b_{y} & 0 & 0 & 0 & b_{\pi } & 1-b_{\pi }%
\end{bmatrix}%
}_{\mathbf{A}}\underbrace{%
\begin{bmatrix}
y_{t-1} \\
y_{t-2} \\
r_{t-1} \\
r_{t-2} \\
\pi _{t-1} \\
\pi _{t-2,4}%
\end{bmatrix}%
}_{\mathbf{x}_{t}} \\
& +\underbrace{%
\begin{bmatrix}
1 & -a_{y,1} & -a_{y,2} & -4\frac{a_{r}}{2} & -4\frac{a_{r}}{2} & -\frac{%
a_{r}}{2} & -\frac{a_{r}}{2} \\
0 & -b_{y} & 0 & 0 & 0 & 0 & 0%
\end{bmatrix}%
}_{\mathbf{H}}\underbrace{%
\begin{bmatrix}
y_{t}^{\ast } \\
y_{t-1}^{\ast } \\
y_{t-2}^{\ast } \\
g_{t-1} \\
g_{t-2} \\
z_{t-1} \\
z_{t-2}%
\end{bmatrix}%
}_{\boldsymbol{\xi }_{t}}+\underbrace{%
\begin{bmatrix}
\sigma _{\tilde{y}} & 0 \\
0 & \sigma _{\pi }%
\end{bmatrix}%
}_{\boldsymbol{R}^{1/2}}\underbrace{%
\begin{bmatrix}
\varepsilon _{t}^{\tilde{y}} \\
\varepsilon _{t}^{\pi }%
\end{bmatrix}%
}_{\boldsymbol{\varepsilon }_{t}^{\mathbf{y}}}.
\end{align}%
\esq The state equation is:%
\begin{align}
\boldsymbol{\xi }_{t}& =\mathbf{F}\boldsymbol{\xi }_{t-1}+\boldsymbol{Q}%
^{1/2}\boldsymbol{\varepsilon }_{t}^{\boldsymbol{\xi }}  \notag \\[5mm]
\underbrace{%
\begin{bmatrix}
y_{t}^{\ast } \\
y_{t-1}^{\ast } \\
y_{t-2}^{\ast } \\
g_{t-1} \\
g_{t-2} \\
z_{t-1} \\
z_{t-2}%
\end{bmatrix}%
}_{\boldsymbol{\xi }_{t}}& =\underbrace{%
\begin{bmatrix}
1 & 0 & 0 & 1 & 0 & 0 & 0 \\
1 & 0 & 0 & 0 & 0 & 0 & 0 \\
0 & 1 & 0 & 0 & 0 & 0 & 0 \\
0 & 0 & 0 & 1 & 0 & 0 & 0 \\
0 & 0 & 0 & 1 & 0 & 0 & 0 \\
0 & 0 & 0 & 0 & 0 & 1 & 0 \\
0 & 0 & 0 & 0 & 0 & 1 & 0%
\end{bmatrix}%
}_{\mathbf{F}}\underbrace{%
\begin{bmatrix}
y_{t-1}^{\ast } \\
y_{t-2}^{\ast } \\
y_{t-3}^{\ast } \\
g_{t-2} \\
g_{t-3} \\
z_{t-2} \\
z_{t-3}%
\end{bmatrix}%
}_{\boldsymbol{\xi }_{t-1}}+\underbrace{%
\begin{bmatrix}
\sigma _{y^{\ast }} & \sigma _{g} & 0 \\
0 & 0 & 0 \\
0 & 0 & 0 \\
0 & \sigma _{g} & 0 \\
0 & 0 & 0 \\
0 & 0 & \sigma _{z} \\
0 & 0 & 0%
\end{bmatrix}%
}_{\boldsymbol{Q}^{1/2}}\underbrace{%
\begin{bmatrix}
\varepsilon _{t}^{y^{\ast }} \\
\varepsilon _{t-1}^{g} \\
\varepsilon _{t-1}^{z}%
\end{bmatrix}%
}_{\boldsymbol{\varepsilon }_{t}^{\boldsymbol{\xi }}}  \label{state1}
\end{align}

\pagebreak

Expanding the relations in \ref{obs0} and \ref{state1} and re-arranging
yields:%
\begin{equation}
\begin{bmatrix}
y_{t} \\
\\
\\
\pi _{t} \\
\end{bmatrix}%
=%
\begin{bmatrix}
y_{t}^{\ast }+a_{y,1}\left( y_{t-1}-y_{t-1}^{\ast }\right) +a_{y,2}\left(
y_{t-2}-y_{t-2}^{\ast }\right)  \\
+\frac{1}{2}a_{r}\left( \left[ r_{t-1}-4cg_{t-1}-z_{t-1}\right] +\left[
r_{t-2}-4cg_{t-2}-z_{t-2}\right] \right) +\sigma _{\tilde{y}}\varepsilon
_{t}^{\tilde{y}} \\
\\
b_{y}\left( y_{t-1}-y_{t-1}^{\ast }\right) +b_{\pi }\pi _{t-1}+(1-b_{\pi
})\pi _{t-2,4}+\sigma _{\pi }\varepsilon _{t}^{\pi }%
\end{bmatrix}
\label{lwa}
\end{equation}

\begin{equation}
\begin{bmatrix}
y_{t}^{\ast } \\
y_{t-1}^{\ast } \\
y_{t-2}^{\ast } \\
g_{t-1} \\
g_{t-2} \\
z_{t-1} \\
z_{t-2}%
\end{bmatrix}%
=%
\begin{bmatrix}
y_{t-1}^{\ast }+\overbrace{g_{t-2}+\sigma _{g}\varepsilon _{t-1}^{g}}%
^{g_{t-1}}+\sigma _{y^{\ast }}\varepsilon _{t}^{y^{\ast }} \\
y_{t-1}^{\ast } \\
y_{t-2}^{\ast } \\
g_{t-2}+\sigma _{g}\varepsilon _{t-1}^{g} \\
g_{t-2} \\
z_{t-2}+\sigma _{z}\varepsilon _{t-1}^{z} \\
z_{t-2}%
\end{bmatrix}%
\Rightarrow
\begin{bmatrix}
\Delta y_{t}^{\ast } \\
\Delta g_{t-1} \\
\Delta z_{t-1}%
\end{bmatrix}%
=%
\begin{bmatrix}
g_{t-1}+\sigma _{y^{\ast }}\varepsilon _{t}^{y^{\ast }} \\
\sigma _{g}\varepsilon _{t-1}^{g} \\
\sigma _{z}\varepsilon _{t-1}^{z}%
\end{bmatrix}
\label{lwb}
\end{equation}%
and with $r_{t}^{\ast }=4g_{t}+z_{t}$, we get: \vsp{-4}
\begin{align}
\Delta r_{t}^{\ast }& =4\overbrace{\Delta g_{t}}^{\sigma _{g}\varepsilon
_{t}^{g}}+\overbrace{\Delta z_{t}}^{\sigma _{z}\varepsilon _{t}^{z}}  \notag
\\
& =4\sigma _{g}\varepsilon _{t}^{g}+\sigma _{z}\varepsilon _{t}^{z}.
\label{drstar}
\end{align}

\section{Shock recovery SSM}

\subsection{SSM with lagged states}

Kurz's (2018) SSM has the following general from:\bsq\label{SSM}%
\begin{align}
\mathsf{Measurement}& :\quad Z_{t}=D_{1}X_{t}+D_{2}X_{t-1}+R\varepsilon _{t}
\label{ssm1} \\
\mathsf{State}& :\quad X_{t}=AX_{t-1}+C\varepsilon _{t},  \label{ssm2}
\end{align}%
\esq where $\varepsilon _{t}\sim MN(0,I_{m})$, $D_{1},D_{2},A,R$ are $C$ are
conformable system matrices, $Z_{t}$ the observed variable and $X_{t}$ the
latent state variable.

\subsection{HLW17 equations}

Following the same format as for LW03, HLW17's SSM equations in \ref{lwa}
and \ref{lwb} (with original labelling of the shocks) gives the following
SSM equations:

\pagebreak \bsq\label{LW03}%
\begin{align}
y_{t}& =y_{t}^{\ast }+\sum_{i=1}^{2}a_{y,i}\left( y_{t-i}-y_{t-i}^{\ast
}\right) +\tfrac{1}{2}a_{r}\sum_{i=1}^{2}\left( r_{t-i}-r_{t-i}^{\ast
}\right) +\sigma _{\tilde{y}}\varepsilon _{t}^{\tilde{y}}  \label{LW03a} \\
\pi _{t}& =b_{y}\left( y_{t-1}-y_{t-1}^{\ast }\right) +b_{\pi }\pi
_{t-1}+(1-b_{\pi })\pi _{t-2,4}+\sigma _{\pi }\varepsilon _{t}^{\pi } \\
\Delta z_{t}& =\sigma _{z}\varepsilon _{t}^{z}  \label{LW03c} \\
\Delta y_{t}^{\ast }& =g_{t-1}+\sigma _{y^{\ast }}\varepsilon _{t}^{y^{\ast
}}  \label{LW03d} \\
\Delta g_{t}& =\sigma _{g}\varepsilon _{t}^{g}  \label{LW03e} \\
\intxt{with}\Delta r_{t}^{\ast }& =4\sigma _{g}\varepsilon _{t}^{g}+\sigma
_{z}\varepsilon _{t}^{z}.  \label{LW03f}
\end{align}%
\esq

\subsection{HLW17 SSM for shock recovery}

To assess recovery, re-write the model in `\emph{shock recovery}' form. That
is, collect all observables in $Z_{t}$, and all shocks (and other state
variables)\ in state vector $X_{t}$. The relevant equations from \ref{LW03}
(incorporating \ref{LW03f}) for the `\emph{shock recovery}' SSM\ are\ (note
the inclusion of $c$ in \ref{drstar2}):\bsq\label{ssm0}%
\begin{align}
\mathsf{Measurement}& :\;\quad Z_{1t}=y_{t}^{\ast }-a_{y,1}y_{t-1}^{\ast
}-a_{y,2}y_{t-2}^{\ast }-\tfrac{1}{2}a_{r}\left( r_{t-1}^{\ast
}+r_{t-2}^{\ast }\right) +\sigma _{\tilde{y}}\varepsilon _{t}^{\tilde{y}} \\
\phantom{\mathsf{Measurement}}& \,\,\;\;\phantom{:\quad}%
Z_{2t}=-b_{y}y_{t-1}^{\ast }+\sigma _{\pi }\varepsilon _{t}^{\pi } \\
\mathsf{State}& :\quad \Delta y_{t}^{\ast }=g_{t-1}+\sigma _{y^{\ast
}}\varepsilon _{t}^{y^{\ast }} \\
\phantom{\mathsf{State}}& \;\,\;\phantom{:\quad}\Delta g_{t}=\sigma
_{g}\varepsilon _{t}^{g} \\
\phantom{\mathsf{State}}& \;\,\;\phantom{:\quad}\Delta r_{t}^{\ast
}=4c\sigma _{g}\varepsilon _{t}^{g}+\sigma _{z}\varepsilon _{t}^{z},
\label{drstar2}
\end{align}%
\esq with the observables $Z_{t}$ in the measurement equations defined as:%
\vsp{-3}
\begin{align*}
Z_{1t}& =y_{t}-\sum_{i=1}^{2}a_{y,i}y_{t-i}-\tfrac{1}{2}a_{r}%
\sum_{i=1}^{2}r_{t-i} \\
Z_{2t}& =\pi _{t}-b_{\pi }\pi _{t-1}-(1-b_{\pi })\pi _{t-2,4}-b_{y}y_{t-1}.
\end{align*}

\pagebreak The `\emph{shock recovery}' SSF\ corresponding to \ref{ssm0} is
then: \bsq\label{K0SSM}%
\begin{align}
\mathsf{Measurement}:\quad Z_{t}& =D_{1}X_{t}+D_{2}X_{t-1}+R\varepsilon _{t}
\notag \\[-4mm]
\underbrace{\left[
\begin{array}{c}
Z_{1t} \\
Z_{2t}%
\end{array}%
\right] }_{Z_{t}}& =\underbrace{%
\begin{bmatrix}
1 & 0 & 0 & 0 & 0 & \sigma _{\tilde{y}} & 0 & 0 & 0 & 0 \\
0 & -b_{y} & 0 & 0 & 0 & 0 & \sigma _{\pi } & 0 & 0 & 0%
\end{bmatrix}%
}_{D_{1}}\underbrace{\left[
\begin{array}{c}
y_{t}^{\ast } \\
y_{t-1}^{\ast } \\
g_{t} \\
r_{t}^{\ast } \\
r_{t-1}^{\ast } \\
\varepsilon _{t}^{\tilde{y}} \\
\varepsilon _{t}^{\pi } \\
\varepsilon _{t}^{z} \\
\varepsilon _{t}^{y^{\ast }} \\
\varepsilon _{t}^{g}%
\end{array}%
\right] }_{X_{t}} \\
& +\underbrace{%
\begin{bmatrix}
-a_{y,1} & -a_{y,2} & 0 & -\frac{a_{r}}{2} & -\frac{a_{r}}{2} & 0 & 0 & 0 & 0
& 0 \\
0 & 0 & 0 & 0 & 0 & 0 & 0 & 0 & 0 & 0%
\end{bmatrix}%
}_{D_{2}}\underbrace{\left[
\begin{array}{c}
y_{t-1}^{\ast } \\
y_{t-2}^{\ast } \\
g_{t-1} \\
r_{t-1}^{\ast } \\
r_{t-2}^{\ast } \\
\varepsilon _{t-1}^{\tilde{y}} \\
\varepsilon _{t-1}^{\pi } \\
\varepsilon _{t-1}^{z} \\
\varepsilon _{t-1}^{y^{\ast }} \\
\varepsilon _{t-1}^{g}%
\end{array}%
\right] }_{X_{t-1}}+\underbrace{\boldsymbol{0}_{2\times 5}}_{R}\underbrace{%
\begin{bmatrix}
\varepsilon _{t}^{\tilde{y}} \\
\varepsilon _{t}^{\pi } \\
\varepsilon _{t}^{z} \\
\varepsilon _{t}^{y^{\ast }} \\
\varepsilon _{t}^{g}%
\end{bmatrix}%
}_{\varepsilon _{t}} \\
\mathsf{State}:\quad X_{t}& =AX_{t-1}+C\varepsilon _{t},  \notag \\[5mm]
\underbrace{\left[
\begin{array}{c}
y_{t}^{\ast } \\
y_{t-1}^{\ast } \\
g_{t} \\
r_{t}^{\ast } \\
r_{t-1}^{\ast } \\
\varepsilon _{t}^{\tilde{y}} \\
\varepsilon _{t}^{\pi } \\
\varepsilon _{t}^{z} \\
\varepsilon _{t}^{y^{\ast }} \\
\varepsilon _{t}^{g}%
\end{array}%
\right] }_{X_{t}}& =\underbrace{%
\begin{bmatrix}
1 & 0 & 1 & 0 & 0 & 0 & 0 & 0 & 0 & 0 \\
1 & 0 & 0 & 0 & 0 & 0 & 0 & 0 & 0 & 0 \\
0 & 0 & 1 & 0 & 0 & 0 & 0 & 0 & 0 & 0 \\
0 & 0 & 0 & 1 & 0 & 0 & 0 & 0 & 0 & 0 \\
0 & 0 & 0 & 1 & 0 & 0 & 0 & 0 & 0 & 0 \\
0 & 0 & 0 & 0 & 0 & 0 & 0 & 0 & 0 & 0 \\
0 & 0 & 0 & 0 & 0 & 0 & 0 & 0 & 0 & 0 \\
0 & 0 & 0 & 0 & 0 & 0 & 0 & 0 & 0 & 0 \\
0 & 0 & 0 & 0 & 0 & 0 & 0 & 0 & 0 & 0 \\
0 & 0 & 0 & 0 & 0 & 0 & 0 & 0 & 0 & 0%
\end{bmatrix}%
}_{A}\underbrace{\left[
\begin{array}{c}
y_{t-1}^{\ast } \\
y_{t-2}^{\ast } \\
g_{t-1} \\
r_{t-1}^{\ast } \\
r_{t-2}^{\ast } \\
\varepsilon _{t-1}^{\tilde{y}} \\
\varepsilon _{t-1}^{\pi } \\
\varepsilon _{t-1}^{z} \\
\varepsilon _{t-1}^{y^{\ast }} \\
\varepsilon _{t-1}^{g}%
\end{array}%
\right] }_{X_{t-1}}+\underbrace{%
\begin{bmatrix}
0 & 0 & 0 & \sigma _{y^{\ast }} & 0 \\
0 & 0 & 0 & 0 & 0 \\
0 & 0 & 0 & 0 & \sigma _{g} \\
0 & 0 & \sigma _{z} & 0 & 4c\sigma _{g} \\
0 & 0 & 0 & 0 & 0 \\
1 & 0 & 0 & 0 & 0 \\
0 & 1 & 0 & 0 & 0 \\
0 & 0 & 1 & 0 & 0 \\
0 & 0 & 0 & 1 & 0 \\
0 & 0 & 0 & 0 & 1%
\end{bmatrix}%
}_{C}\underbrace{%
\begin{bmatrix}
\varepsilon _{t}^{\tilde{y}} \\
\varepsilon _{t}^{\pi } \\
\varepsilon _{t}^{z} \\
\varepsilon _{t}^{y^{\ast }} \\
\varepsilon _{t}^{g}%
\end{bmatrix}%
}_{\varepsilon _{t}}
\end{align}%
\esq

\subsubsection{Correlation between true change in natural rate and estimate}

The correlation between the true and estimated $\Delta r_{t}^{\ast }$ from
the SSM can be constructed from the relation:%
\begin{equation}
\rho =0.5\frac{\mathrm{Var}(\Delta r_{t}^{\ast })+\mathrm{Var}(E_{T}\Delta
r_{t}^{\ast })-\phi }{\sigma (\Delta r_{t}^{\ast })\sigma (E_{T}\Delta
r_{t}^{\ast })},
\end{equation}
where $\mathrm{Var}(\Delta r_{t}^{\ast })=4^{2}c^{2}\sigma _{5}^{2}+\sigma
_{3}^{2}$, $\sigma (\Delta r_{t}^{\ast })=\sqrt{\mathrm{Var}(\Delta
r_{t}^{\ast })}$, and $\mathrm{Var}(E_{T}\Delta r_{t}^{\ast })$ can be
computed from simulating from the true model, applying the Kalman Filter and
Smoother to get $E_{T}\Delta r_{t}^{\ast }$ and then computing the sample
variance of $E_{T}\Delta r_{t}^{\ast }$ as an estimate of $\mathrm{Var}%
(E_{T}\Delta r_{t}^{\ast })$.

To obtain $\phi $, add $\Delta r_{t}^{\ast }$ to the state-vector $X_{t}$
and augment the remaining matrices to be conformable. The required $\phi $
term is then the entry of $\mathrm{diag}(P_{t|T}^{\ast })$ that corresponds
to $\Delta r_{t}^{\ast }$, which will be the very last element.

The augmented SSF of \ref{K0SSM} is then:
\begin{align}
\mathsf{Measurement}:\quad Z_{t}& =D_{1}X_{t}+D_{2}X_{t-1}+R\varepsilon _{t}
\notag \\[-4mm]
\underbrace{\left[
\begin{array}{c}
Z_{1t} \\
Z_{2t}%
\end{array}%
\right] }_{Z_{t}}& =\underbrace{%
\begin{bmatrix}
1 & 0 & 0 & 0 & 0 & \sigma _{\tilde{y}} & 0 & 0 & 0 & 0 & \color{red}{0} \\
0 & -b_{y} & 0 & 0 & 0 & 0 & \sigma _{\pi } & 0 & 0 & 0 & \color{red}{0}%
\end{bmatrix}%
}_{D_{1}}\underbrace{\left[
\begin{array}{c}
y_{t}^{\ast } \\
y_{t-1}^{\ast } \\
g_{t} \\
r_{t}^{\ast } \\
r_{t-1}^{\ast } \\
\varepsilon _{t}^{\tilde{y}} \\
\varepsilon _{t}^{\pi } \\
\varepsilon _{t}^{z} \\
\varepsilon _{t}^{y^{\ast }} \\
\varepsilon _{t}^{g} \\
\color{red}{\Delta r_t^*}%
\end{array}%
\right] }_{X_{t}} \\
& +\underbrace{%
\begin{bmatrix}
-a_{y,1} & -a_{y,2} & 0 & -\frac{a_{r}}{2} & -\frac{a_{r}}{2} & 0 & 0 & 0 & 0
& 0 & \color{red}{0} \\
0 & 0 & 0 & 0 & 0 & 0 & 0 & 0 & 0 & 0 & \color{red}{0}%
\end{bmatrix}%
}_{D_{2}}\underbrace{\left[
\begin{array}{c}
y_{t-1}^{\ast } \\
y_{t-2}^{\ast } \\
g_{t-1} \\
r_{t-1}^{\ast } \\
r_{t-2}^{\ast } \\
\varepsilon _{t-1}^{\tilde{y}} \\
\varepsilon _{t-1}^{\pi } \\
\varepsilon _{t-1}^{z} \\
\varepsilon _{t-1}^{y^{\ast }} \\
\varepsilon _{t-1}^{g} \\
\color{red}{\Delta r_{t-1}^*}%
\end{array}%
\right] }_{X_{t-1}}+\underbrace{\boldsymbol{0}_{2\times 5}}_{R}\underbrace{%
\begin{bmatrix}
\varepsilon _{t}^{\tilde{y}} \\
\varepsilon _{t}^{\pi } \\
\varepsilon _{t}^{z} \\
\varepsilon _{t}^{y^{\ast }} \\
\varepsilon _{t}^{g}%
\end{bmatrix}%
}_{\varepsilon _{t}} \\
\mathsf{State}:\quad X_{t}& =AX_{t-1}+C\varepsilon _{t},  \notag \\[5mm]
\underbrace{\left[
\begin{array}{c}
y_{t}^{\ast } \\
y_{t-1}^{\ast } \\
g_{t} \\
r_{t}^{\ast } \\
r_{t-1}^{\ast } \\
\varepsilon _{t}^{\tilde{y}} \\
\varepsilon _{t}^{\pi } \\
\varepsilon _{t}^{z} \\
\varepsilon _{t}^{y^{\ast }} \\
\varepsilon _{t}^{g} \\
\color{red}{\Delta r_t^*}%
\end{array}%
\right] }_{X_{t}}& =\underbrace{%
\begin{bmatrix}
1 & 0 & 1 & 0 & 0 & 0 & 0 & 0 & 0 & 0 & \color{red}{0} \\
1 & 0 & 0 & 0 & 0 & 0 & 0 & 0 & 0 & 0 & \color{red}{0} \\
0 & 0 & 1 & 0 & 0 & 0 & 0 & 0 & 0 & 0 & \color{red}{0} \\
0 & 0 & 0 & 1 & 0 & 0 & 0 & 0 & 0 & 0 & \color{red}{0} \\
0 & 0 & 0 & 1 & 0 & 0 & 0 & 0 & 0 & 0 & \color{red}{0} \\
0 & 0 & 0 & 0 & 0 & 0 & 0 & 0 & 0 & 0 & \color{red}{0} \\
0 & 0 & 0 & 0 & 0 & 0 & 0 & 0 & 0 & 0 & \color{red}{0} \\
0 & 0 & 0 & 0 & 0 & 0 & 0 & 0 & 0 & 0 & \color{red}{0} \\
0 & 0 & 0 & 0 & 0 & 0 & 0 & 0 & 0 & 0 & \color{red}{0} \\
0 & 0 & 0 & 0 & 0 & 0 & 0 & 0 & 0 & 0 & \color{red}{0} \\
\color{red}{0} & \color{red}{0} & \color{red}{0} & \color{red}{0} & %
\color{red}{0} & \color{red}{0} & \color{red}{0} & \color{red}{0} & %
\color{red}{0} & \color{red}{0} & \color{red}{0}%
\end{bmatrix}%
}_{A}\underbrace{\left[
\begin{array}{c}
y_{t-1}^{\ast } \\
y_{t-2}^{\ast } \\
g_{t-1} \\
r_{t-1}^{\ast } \\
r_{t-2}^{\ast } \\
\varepsilon _{t-1}^{\tilde{y}} \\
\varepsilon _{t-1}^{\pi } \\
\varepsilon _{t-1}^{z} \\
\varepsilon _{t-1}^{y^{\ast }} \\
\varepsilon _{t-1}^{g} \\
\color{red}{\Delta r_{t-1}^*}%
\end{array}%
\right] }_{X_{t-1}}+\underbrace{%
\begin{bmatrix}
0 & 0 & 0 & \sigma _{y^{\ast }} & 0 \\
0 & 0 & 0 & 0 & 0 \\
0 & 0 & 0 & 0 & \sigma _{g} \\
0 & 0 & \sigma _{z} & 0 & 4c\sigma _{g} \\
0 & 0 & 0 & 0 & 0 \\
1 & 0 & 0 & 0 & 0 \\
0 & 1 & 0 & 0 & 0 \\
0 & 0 & 1 & 0 & 0 \\
0 & 0 & 0 & 1 & 0 \\
0 & 0 & 0 & 0 & 1 \\
\color{red}{0} & \color{red}{0} & \color{red}{\sigma _{z}} & \color{red}{0}
& \color{red}{4c\sigma _{g}}%
\end{bmatrix}%
}_{C}\underbrace{%
\begin{bmatrix}
\varepsilon _{t}^{\tilde{y}} \\
\varepsilon _{t}^{\pi } \\
\varepsilon _{t}^{z} \\
\varepsilon _{t}^{y^{\ast }} \\
\varepsilon _{t}^{g}%
\end{bmatrix}%
}_{\varepsilon _{t}}
\end{align}

\end{document}
