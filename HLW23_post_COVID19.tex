
\documentclass[a4paper,12pt]{article}
%%%%%%%%%%%%%%%%%%%%%%%%%%%%%%%%%%%%%%%%%%%%%%%%%%%%%%%%%%%%%%%%%%%%%%%%%%%%%%%%%%%%%%%%%%%%%%%%%%%%%%%%%%%%%%%%%%%%%%%%%%%%%%%%%%%%%%%%%%%%%%%%%%%%%%%%%%%%%%%%%%%%%%%%%%%%%%%%%%%%%%%%%%%%%%%%%%%%%%%%%%%%%%%%%%%%%%%%%%%%%%%%%%%%%%%%%%%%%%%%%%%%%%%%%%%%
\usepackage{graphicx,hyperref,mathpple,amsmath,exscale,setspace,xcolor}
\usepackage[left=20mm,right=20mm,top=20mm,bottom=20mm]{geometry}
\usepackage{pdflscape,showkeys,changepage}
\usepackage[round]{natbib}

\setcounter{MaxMatrixCols}{11}
%TCIDATA{OutputFilter=LATEX.DLL}
%TCIDATA{Version=5.50.0.2953}
%TCIDATA{<META NAME="SaveForMode" CONTENT="2">}
%TCIDATA{BibliographyScheme=BibTeX}
%TCIDATA{Created=Wednesday, May 03, 2023 13:45:06}
%TCIDATA{LastRevised=Wednesday, October 25, 2023 14:32:42}
%TCIDATA{<META NAME="GraphicsSave" CONTENT="32">}
%TCIDATA{<META NAME="DocumentShell" CONTENT="Standard LaTeX\Blank - Standard LaTeX Article">}
%TCIDATA{CSTFile=40 LaTeX article.cst}

\let\oldref\ref
\AtBeginDocument{
\let\oldref\ref\renewcommand{\ref}[1]{(\oldref{#1})}
\newcommand{\bsq}{\begin{subequations}}\newcommand{\esq}{\end{subequations}}
\newcommand{\bls}{\begin{landscape}}\newcommand{\els}{\end{landscape}}
\renewcommand\showkeyslabelformat[1]{{\parbox[t]{\marginparwidth}{\raggedright\footnotesize\url{#1}}}}
\newcommand{\intxt}[1]{\intertext{#1}}\newcommand{\BAW}[1]{\begin{adjustwidth}{-#1mm}{-5mm}}\newcommand{\EAW}{\end{adjustwidth}}
\newcommand{\vsp}[1]{\vspace*{#1mm}}\newcommand{\hsp}[1]{\hspace*{#1mm}}  }
\makeatletter
\renewcommand*{\@fnsymbol}[1]{\ensuremath{\ifcase#1\or *\or
    \#\or \star\or \bowtie\or \star\star\or \ddagger\ddagger \else\@ctrerr\fi}}
\makeatother
\allowdisplaybreaks
\IfFileExists{C:/swp55/TCITeX/TeX/LaTeX/SWmacros/tcilatex.tex}{\input{tcilatex}}{}
\newcommand{\dble}{1.77}
\newcommand{\sngl}{1.23}
\definecolor{myred}{rgb}{.50,.10,.10}
\definecolor{mygrn}{rgb}{.10,.35,.10}
\definecolor{myblu}{rgb}{.10,.10,.35}
\hypersetup{colorlinks,citecolor=myblu,filecolor=mygrn,linkcolor=myred,urlcolor=mygrn,breaklinks=true}
\setstretch{\sngl}

\begin{document}


\section{HLW's (2023) post COVID19 State-Space Model (SSM) Form}

HLW23 (post\ COVID19) use the same standard State-Space Form (SSF) as in
HLW17:%
% (see their \texttt{%
%LW\_Code\_Guide.pdf} file \texttt{LW\_replication.zip} from the NYFED
%website):%
\begin{align*}
\mathsf{Measurement}& :\quad \mathbf{y}_{t}=\mathbf{Ax}_{t}+\mathbf{H}%
\boldsymbol{\xi }_{t}+\boldsymbol{R}_{t}^{1/2}\boldsymbol{\varepsilon }_{t}^{%
\mathbf{y}} \\
\mathsf{State}& :\quad \boldsymbol{\xi }_{t}=\mathbf{F}\boldsymbol{\xi }%
_{t-1}+\boldsymbol{Q}^{1/2}\boldsymbol{\varepsilon }_{t}^{\boldsymbol{\xi }}
\end{align*}%
but where the states and exogenous variables have been modified:%
\begin{align*}
\mathbf{y}_{t}& =%
\begin{bmatrix}
y_{t} & \pi _{t}%
\end{bmatrix}%
^{\prime }, \\
\mathbf{x}_{t}& =%
\begin{bmatrix}
y_{t-1} & y_{t-2} & r_{t-1} & r_{t-2} & \pi _{t-1} & \pi _{t-2,4} & d_{t} & 
d_{t-1} & d_{t-2}%
\end{bmatrix}%
^{\prime }, \\
\boldsymbol{\xi }_{t}& =%
\begin{bmatrix}
y_{t}^{\ast } & y_{t-1}^{\ast } & y_{t-2}^{\ast } & g_{t} & g_{t-1} & g_{t-2}
& z_{t} & z_{t-1} & z_{t-2}%
\end{bmatrix}%
^{\prime }, \\
\mathbf{A}& =%
\begin{bmatrix}
a_{y,1} & a_{y,2} & \frac{a_{r}}{2} & \frac{a_{r}}{2} & 0 & 0 & \phi & -\phi
a_{y,1} & -\phi a_{y,2} \\ 
b_{y} & 0 & 0 & 0 & b_{\pi } & 1-b_{\pi } & 0 & -\phi & 0%
\end{bmatrix}%
, \\
\mathbf{H}& =%
\begin{bmatrix}
1 & -a_{y,1} & -a_{y,2} & 0 & -4c\frac{a_{r}}{2} & -4c\frac{a_{r}}{2} & 0 & -%
\frac{a_{r}}{2} & -\frac{a_{r}}{2} \\ 
0 & -b_{y} & 0 & 0 & 0 & 0 & 0 & 0 & 0%
\end{bmatrix}%
.
\end{align*}%
to incorporate dummy variables in the measurement equation and a
time-varying $R_{t}$ covariance matrix, and some correction I\ pointed out
in the formulation of the state equation (see pages $7-8$ in `\texttt{%
HLW\_Replication\_Code\_Guide.pdf}' and pages $43-44$ in `\emph{Measuring
the Natural Rate of Interest after COVID-19}'.

\noindent \textbf{Note:}\ This\ follows the notation used in the
documentation file:\ `\texttt{HLW\_Replication\_Code\_Guide.pdf}' included
in the zip file: `\texttt{HLW\_2023\_Replication\_Code.zip}' that contains
the replication code of HLW23 and which is available from the NYFED\ website
at: %
\url{https://www.newyorkfed.org/medialibrary/media/research/economists/williams/data/HLW_Code.zip}%
.

Compared to HLW17, the constant $c$ is estimated again and dummies are added
to account for the COVID19 years.

In the construction of $r_{t}^{\ast }$, trend growth $g_{t}$ is again
annualized, but not in the state equations for $g_{t}$. That is, in the
matrices above, the entries in $\mathbf{H(}1,4:5\mathbf{)}$ corresponding to
trend growth are multiplied by 4 (as well as $c$) in the code that performs
the estimation (see \texttt{unpack.parameters.stage3.R} in \texttt{%
HLW\_replication.zip} files, line 30, which reads):\ 

\texttt{H[1, 5:6] \TEXTsymbol{<}-
-parameters[param.num["c"]]*parameters[param.num["a\_r"]]*2}.

\noindent The standard deviations of the shocks are denoted by $%
\begin{bmatrix}
\sigma _{\tilde{y}} & \sigma _{\pi } & \sigma _{y^{\ast }} & \sigma _{g} & 
\sigma _{z}%
\end{bmatrix}%
^{\prime }$ in the documentation in `\texttt{HLW\_Code\_Guide.pdf}'.

\pagebreak

\subsection{SSM of HLW23}

The measurement equation is:\bsq\label{obs0} 
\begin{align}
\mathbf{y}_{t}& =\mathbf{Ax}_{t}+\mathbf{H}\xi _{t}+\boldsymbol{R}_{t}^{1/2}%
\boldsymbol{\varepsilon }_{t}^{\mathbf{y}}  \notag \\
\underbrace{%
\begin{bmatrix}
y_{t} \\ 
\pi _{t}%
\end{bmatrix}%
}_{\mathbf{y}_{t}}& =\underbrace{%
\begin{bmatrix}
a_{y,1} & a_{y,2} & \frac{a_{r}}{2} & \frac{a_{r}}{2} & 0 & 0 & \phi & -\phi
a_{y,1} & -\phi a_{y,2} \\ 
b_{y} & 0 & 0 & 0 & b_{\pi } & 1-b_{\pi } & 0 & -\phi b_{y} & 0%
\end{bmatrix}%
}_{\mathbf{A}}\underbrace{%
\begin{bmatrix}
y_{t-1} \\ 
y_{t-2} \\ 
r_{t-1} \\ 
r_{t-2} \\ 
\pi _{t-1} \\ 
\pi _{t-2,4} \\ 
d_{t} \\ 
d_{t-1} \\ 
d_{t-2}%
\end{bmatrix}%
}_{\mathbf{x}_{t}} \\
& +\underbrace{%
\begin{bmatrix}
1 & -a_{y,1} & -a_{y,2} & 0 & -4c\frac{a_{r}}{2} & -4c\frac{a_{r}}{2} & 0 & -%
\frac{a_{r}}{2} & -\frac{a_{r}}{2} \\ 
0 & -b_{y} & 0 & 0 & 0 & 0 & 0 & 0 & 0%
\end{bmatrix}%
}_{\mathbf{H}}\underbrace{%
\begin{bmatrix}
y_{t}^{\ast } \\ 
y_{t-1}^{\ast } \\ 
y_{t-2}^{\ast } \\ 
g_{t} \\ 
g_{t-1} \\ 
g_{t-2} \\ 
z_{t} \\ 
z_{t-1} \\ 
z_{t-2}%
\end{bmatrix}%
}_{\boldsymbol{\xi }_{t}}+\underbrace{%
\begin{bmatrix}
\kappa _{t}\sigma _{\tilde{y}} & 0 \\ 
0 & \kappa _{t}\sigma _{\pi }%
\end{bmatrix}%
}_{\boldsymbol{R}_{t}^{1/2}}\underbrace{%
\begin{bmatrix}
\varepsilon _{t}^{\tilde{y}} \\ 
\varepsilon _{t}^{\pi }%
\end{bmatrix}%
}_{\boldsymbol{\varepsilon }_{t}^{\mathbf{y}}}.
\end{align}%
\esq The state equation is:%
\begin{align}
\boldsymbol{\xi }_{t}& =\mathbf{F}\boldsymbol{\xi }_{t-1}+\boldsymbol{Q}%
^{1/2}\boldsymbol{\varepsilon }_{t}^{\boldsymbol{\xi }}  \notag \\[5mm]
\underbrace{%
\begin{bmatrix}
y_{t}^{\ast } \\ 
y_{t-1}^{\ast } \\ 
y_{t-2}^{\ast } \\ 
g_{t} \\ 
g_{t-1} \\ 
g_{t-2} \\ 
z_{t} \\ 
z_{t-1} \\ 
z_{t-2}%
\end{bmatrix}%
}_{\boldsymbol{\xi }_{t}}& =\underbrace{%
\begin{bmatrix}
1 & 0 & 0 & 1 & 0 & 0 & 0 & 0 & 0 \\ 
1 & 0 & 0 & 0 & 0 & 0 & 0 & 0 & 0 \\ 
0 & 1 & 0 & 0 & 0 & 0 & 0 & 0 & 0 \\ 
0 & 0 & 0 & 1 & 0 & 0 & 0 & 0 & 0 \\ 
0 & 0 & 0 & 1 & 0 & 0 & 0 & 0 & 0 \\ 
0 & 0 & 0 & 0 & 1 & 0 & 0 & 0 & 0 \\ 
0 & 0 & 0 & 0 & 0 & 0 & 1 & 0 & 0 \\ 
0 & 0 & 0 & 0 & 0 & 0 & 1 & 0 & 0 \\ 
0 & 0 & 0 & 0 & 0 & 0 & 0 & 1 & 0%
\end{bmatrix}%
}_{\mathbf{F}}\underbrace{%
\begin{bmatrix}
y_{t-1}^{\ast } \\ 
y_{t-2}^{\ast } \\ 
y_{t-3}^{\ast } \\ 
g_{t-1} \\ 
g_{t-2} \\ 
g_{t-3} \\ 
z_{t-1} \\ 
z_{t-2} \\ 
z_{t-3}%
\end{bmatrix}%
}_{\boldsymbol{\xi }_{t-1}}+\underbrace{%
\begin{bmatrix}
\sigma _{y^{\ast }} & 0 & 0 \\ 
0 & 0 & 0 \\ 
0 & 0 & 0 \\ 
0 & \sigma _{g} & 0 \\ 
0 & 0 & 0 \\ 
0 & 0 & 0 \\ 
0 & 0 & \sigma _{z} \\ 
0 & 0 & 0 \\ 
0 & 0 & 0%
\end{bmatrix}%
}_{\boldsymbol{Q}^{1/2}}\underbrace{%
\begin{bmatrix}
\varepsilon _{t}^{y^{\ast }} \\ 
\varepsilon _{t}^{g} \\ 
\varepsilon _{t}^{z}%
\end{bmatrix}%
}_{\boldsymbol{\varepsilon }_{t}^{\boldsymbol{\xi }}}  \label{state1}
\end{align}

\pagebreak

Expanding the relations in \ref{obs0} and \ref{state1} and re-arranging
yields:%
\begin{equation}
\begin{bmatrix}
y_{t} \\ 
\\ 
\\ 
\pi _{t} \\ 
\end{bmatrix}%
=%
\begin{bmatrix}
y_{t}^{\ast }+\phi d_{t}+a_{y,1}(\overbrace{y_{t-1}-y_{t-1}^{\ast }-\phi
d_{t-1}}^{\tilde{y}_{t-1,COVID}})+a_{y,2}(\overbrace{y_{t-2}-y_{t-2}^{\ast
}-\phi d_{t-2}}^{\tilde{y}_{t-2,COVID}}) \\ 
+\frac{1}{2}a_{r}\left( \left[ r_{t-1}-4cg_{t-1}-z_{t-1}\right] +\left[
r_{t-2}-4cg_{t-2}-z_{t-2}\right] \right) +\kappa _{t}\sigma _{\tilde{y}%
}\varepsilon _{t}^{\tilde{y}} \\ 
\\[-5mm] 
b_{y}(\overbrace{y_{t-1}-y_{t-1}^{\ast }-\phi d_{t-1}}^{\tilde{y}%
_{t-1,COVID}})+b_{\pi }\pi _{t-1}+\left( 1-b_{\pi }1\right) \pi
_{t-2,4}+\kappa _{t}\sigma _{\pi }\kappa _{t}\varepsilon _{t}^{\pi }%
\end{bmatrix}
\label{lwa}
\end{equation}

\begin{equation}
\begin{bmatrix}
y_{t}^{\ast } \\ 
y_{t-1}^{\ast } \\ 
y_{t-2}^{\ast } \\ 
g_{t} \\ 
g_{t-1} \\ 
g_{t-2} \\ 
z_{t} \\ 
z_{t-1} \\ 
z_{t-2}%
\end{bmatrix}%
=%
\begin{bmatrix}
y_{t-1}^{\ast }+g_{t-1}+\sigma _{y^{\ast }}\varepsilon _{t}^{y^{\ast }} \\ 
y_{t-1}^{\ast } \\ 
y_{t-2}^{\ast } \\ 
g_{t-1}+\sigma _{g}\varepsilon _{t}^{g} \\ 
g_{t-1} \\ 
g_{t-2} \\ 
z_{t-1}+\sigma _{z}\varepsilon _{t}^{z} \\ 
z_{t-1} \\ 
z_{t-2}%
\end{bmatrix}%
\Rightarrow 
\begin{bmatrix}
\Delta y_{t}^{\ast } \\ 
\Delta g_{t-1} \\ 
\Delta z_{t-1}%
\end{bmatrix}%
=%
\begin{bmatrix}
g_{t-1}+\sigma _{y^{\ast }}\varepsilon _{t}^{y^{\ast }} \\ 
\sigma _{g}\varepsilon _{t}^{g} \\ 
\sigma _{z}\varepsilon _{t}^{z}%
\end{bmatrix}
\label{lwb}
\end{equation}%
and with $r_{t}^{\ast }=4cg_{t}+z_{t}$, we get: \vsp{-4} 
\begin{align}
\Delta r_{t}^{\ast }& =4\overbrace{c\Delta g_{t}}^{\sigma _{g}\varepsilon
_{t}^{g}}+\overbrace{\Delta z_{t}}^{\sigma _{z}\varepsilon _{t}^{z}}  \notag
\\
& =4c\sigma _{g}\varepsilon _{t}^{g}+\sigma _{z}\varepsilon _{t}^{z}.
\label{drstar}
\end{align}%
Note here that the timing mismatch has been resolved now so that the
equations read as expected, that is, without any lagged values in $%
\boldsymbol{\varepsilon }_{t}^{\boldsymbol{\xi }}$ in \ref{state1}.

The COVID-adjusted natural rate of output is defined in equation 16 of their
paper: 
\begin{equation*}
\tilde{y}_{t,COVID}=100(y_{t}-y_{t,COVID}^{\ast })=100\left(
y_{t}-y_{t}^{\ast }\right) -\phi d_{t}.
\end{equation*}

\section{Shock recovery SSM}

\subsection{SSM with lagged states}

Kurz's (2018) SSM has the following general from:\bsq\label{SSM}%
\begin{align}
\mathsf{Measurement}& :\quad Z_{t}=D_{1}X_{t}+D_{2}X_{t-1}+R\varepsilon _{t}
\label{ssm1} \\
\mathsf{State}& :\quad X_{t}=AX_{t-1}+C\varepsilon _{t},  \label{ssm2}
\end{align}%
\esq where $\varepsilon _{t}\sim MN(0,I_{m})$, $D_{1},D_{2},A,R$ are $C$ are
conformable system matrices, $Z_{t}$ the observed variable and $X_{t}$ the
latent state variable.

\subsection{HLW23 equations}

Following the same format as before, HLW23's SSM equations in \ref{lwa} and %
\ref{lwb} gives the following SSM equations:

\bsq\label{LW03}%
\begin{align}
y_{t}& =y_{t}^{\ast }+\phi d_{t}+\sum_{i=1}^{2}a_{y,i}\left(
y_{t-i}-y_{t-i}^{\ast }-\phi d_{t-i}\right) +\tfrac{1}{2}a_{r}\sum_{i=1}^{2}%
\left( r_{t-i}-r_{t-i}^{\ast }\right) +\kappa _{t}\sigma _{\tilde{y}%
}\varepsilon _{t}^{\tilde{y}}  \label{LW03a} \\
\pi _{t}& =b_{y}(y_{t-1}-y_{t-1}^{\ast }-\phi d_{t-1})+b_{\pi }\pi
_{t-1}+(1-b_{\pi })\pi _{t-2,4}+\kappa _{t}\sigma _{\pi }\varepsilon
_{t}^{\pi } \\
\Delta z_{t}& =\sigma _{z}\varepsilon _{t}^{z}  \label{LW03c} \\
\Delta y_{t}^{\ast }& =g_{t-1}+\sigma _{y^{\ast }}\varepsilon _{t}^{y^{\ast
}}  \label{LW03d} \\
\Delta g_{t}& =\sigma _{g}\varepsilon _{t}^{g}  \label{LW03e} \\
\intxt{with}\Delta r_{t}^{\ast }& =4c\sigma _{g}\varepsilon _{t}^{g}+\sigma
_{z}\varepsilon _{t}^{z}.  \label{LW03f}
\end{align}%
\esq where the numbered shock to named shock mapping is:%
\begin{equation}
\begin{bmatrix}
\varepsilon _{1t} \\ 
\varepsilon _{2t} \\ 
\varepsilon _{3t} \\ 
\varepsilon _{4t} \\ 
\varepsilon _{5t}%
\end{bmatrix}%
=%
\begin{bmatrix}
\varepsilon _{t}^{\tilde{y}} \\ 
\varepsilon _{t}^{\pi } \\ 
\varepsilon _{t}^{z} \\ 
\varepsilon _{t}^{y^{\ast }} \\ 
\varepsilon _{t}^{g}%
\end{bmatrix}%
.
\end{equation}

\subsection{HLW23 SSM for shock recovery}

To assess recovery, re-write the model in `\emph{shock recovery}' form. That
is, collect all observables in $Z_{t}$, and all shocks (and other state
variables)\ in state vector $X_{t}$. The relevant equations from \ref{LW03}
(incorporating \ref{LW03f}) for the `\emph{shock recovery}' SSM\ are\ (note
the inclusion of $c$ in \ref{drstar2}):\bsq\label{ssm0}%
\begin{align}
\mathsf{Measurement}& :\;\quad Z_{1t}=y_{t}^{\ast }-a_{y,1}y_{t-1}^{\ast
}-a_{y,2}y_{t-2}^{\ast }-\tfrac{1}{2}a_{r}\left( r_{t-1}^{\ast
}+r_{t-2}^{\ast }\right) +\kappa _{t}\sigma _{\tilde{y}}\varepsilon _{t}^{%
\tilde{y}} \\
\phantom{\mathsf{Measurement}}& \,\,\;\;\phantom{:\quad}%
Z_{2t}=-b_{y}y_{t-1}^{\ast }+\kappa _{t}\sigma _{\pi }\varepsilon _{t}^{\pi }
\\
\mathsf{State}& :\quad \Delta y_{t}^{\ast }=g_{t-1}+\sigma _{y^{\ast
}}\varepsilon _{t}^{y^{\ast }} \\
\phantom{\mathsf{State}}& \;\,\;\phantom{:\quad}\Delta g_{t}=\sigma
_{g}\varepsilon _{t}^{g} \\
\phantom{\mathsf{State}}& \;\,\;\phantom{:\quad}\Delta r_{t}^{\ast
}=4c\sigma _{g}\varepsilon _{t}^{g}+\sigma _{z}\varepsilon _{t}^{z},
\label{drstar2}
\end{align}%
\esq with the observables $Z_{t}$ in the measurement equations defined as:%
\vsp{-3}\bsq\label{ssmO} 
\begin{align}
Z_{1t}& =y_{t}-\sum_{i=1}^{2}(a_{y,i}y_{t-i}-\phi d_{t-i})-\tfrac{1}{2}%
a_{r}\sum_{i=1}^{2}r_{t-i} \\
Z_{2t}& =\pi _{t}-b_{\pi }\pi _{t-1}-(1-b_{\pi })\pi
_{t-2,4}-(b_{y}y_{t-1}-\phi d_{t-1}).
\end{align}%
\esq So these are the same as before for the HLW17 model, with the dummy
variables only impacting the observable part of the $Z_{t}$ measurement in %
\ref{ssmO} and the only change being the inclusion of the 3 different
possibilities for $\kappa _{t}$: $\kappa _{20220Q2-Q4},\kappa _{2021},\kappa
_{2022}$ and 1 for the baseline, but with the 2023 estimates of all other
parameters.

The `\emph{shock recovery}' SSF\ corresponding to \ref{ssm0} is then: \bsq%
\label{K0SSM}%
\begin{align}
\mathsf{Measurement}:\quad Z_{t}& =D_{1}X_{t}+D_{2}X_{t-1}+R\varepsilon _{t}
\notag \\[-24mm]
\underbrace{\left[ 
\begin{array}{c}
Z_{1t} \\ 
Z_{2t}%
\end{array}%
\right] }_{Z_{t}}& =\underbrace{%
\begin{bmatrix}
1 & 0 & 0 & 0 & 0 & \kappa _{t}\sigma _{\tilde{y}} & 0 & 0 & 0 & 0 \\ 
0 & -b_{y} & 0 & 0 & 0 & 0 & \kappa _{t}\sigma _{\pi } & 0 & 0 & 0%
\end{bmatrix}%
}_{D_{1}}\underbrace{\left[ 
\begin{array}{c}
y_{t}^{\ast } \\ 
y_{t-1}^{\ast } \\ 
g_{t} \\ 
r_{t}^{\ast } \\ 
r_{t-1}^{\ast } \\ 
\varepsilon _{t}^{\tilde{y}} \\ 
\varepsilon _{t}^{\pi } \\ 
\varepsilon _{t}^{z} \\ 
\varepsilon _{t}^{y^{\ast }} \\ 
\varepsilon _{t}^{g}%
\end{array}%
\right] }_{X_{t}} \\
& +\underbrace{%
\begin{bmatrix}
-a_{y,1} & -a_{y,2} & 0 & -\frac{a_{r}}{2} & -\frac{a_{r}}{2} & 0 & 0 & 0 & 0
& 0 \\ 
0 & 0 & 0 & 0 & 0 & 0 & 0 & 0 & 0 & 0%
\end{bmatrix}%
}_{D_{2}}\underbrace{\left[ 
\begin{array}{c}
y_{t-1}^{\ast } \\ 
y_{t-2}^{\ast } \\ 
g_{t-1} \\ 
r_{t-1}^{\ast } \\ 
r_{t-2}^{\ast } \\ 
\varepsilon _{t-1}^{\tilde{y}} \\ 
\varepsilon _{t-1}^{\pi } \\ 
\varepsilon _{t-1}^{z} \\ 
\varepsilon _{t-1}^{y^{\ast }} \\ 
\varepsilon _{t-1}^{g}%
\end{array}%
\right] }_{X_{t-1}}+\underbrace{\boldsymbol{0}_{2\times 5}}_{R}\underbrace{%
\begin{bmatrix}
\varepsilon _{t}^{\tilde{y}} \\ 
\varepsilon _{t}^{\pi } \\ 
\varepsilon _{t}^{z} \\ 
\varepsilon _{t}^{y^{\ast }} \\ 
\varepsilon _{t}^{g}%
\end{bmatrix}%
}_{\varepsilon _{t}} \\
\mathsf{State}:\quad X_{t}& =AX_{t-1}+C\varepsilon _{t},  \notag \\[5mm]
\underbrace{\left[ 
\begin{array}{c}
y_{t}^{\ast } \\ 
y_{t-1}^{\ast } \\ 
g_{t} \\ 
r_{t}^{\ast } \\ 
r_{t-1}^{\ast } \\ 
\varepsilon _{t}^{\tilde{y}} \\ 
\varepsilon _{t}^{\pi } \\ 
\varepsilon _{t}^{z} \\ 
\varepsilon _{t}^{y^{\ast }} \\ 
\varepsilon _{t}^{g}%
\end{array}%
\right] }_{X_{t}}& =\underbrace{%
\begin{bmatrix}
1 & 0 & 1 & 0 & 0 & 0 & 0 & 0 & 0 & 0 \\ 
1 & 0 & 0 & 0 & 0 & 0 & 0 & 0 & 0 & 0 \\ 
0 & 0 & 1 & 0 & 0 & 0 & 0 & 0 & 0 & 0 \\ 
0 & 0 & 0 & 1 & 0 & 0 & 0 & 0 & 0 & 0 \\ 
0 & 0 & 0 & 1 & 0 & 0 & 0 & 0 & 0 & 0 \\ 
0 & 0 & 0 & 0 & 0 & 0 & 0 & 0 & 0 & 0 \\ 
0 & 0 & 0 & 0 & 0 & 0 & 0 & 0 & 0 & 0 \\ 
0 & 0 & 0 & 0 & 0 & 0 & 0 & 0 & 0 & 0 \\ 
0 & 0 & 0 & 0 & 0 & 0 & 0 & 0 & 0 & 0 \\ 
0 & 0 & 0 & 0 & 0 & 0 & 0 & 0 & 0 & 0%
\end{bmatrix}%
}_{A}\underbrace{\left[ 
\begin{array}{c}
y_{t-1}^{\ast } \\ 
y_{t-2}^{\ast } \\ 
g_{t-1} \\ 
r_{t-1}^{\ast } \\ 
r_{t-2}^{\ast } \\ 
\varepsilon _{t-1}^{\tilde{y}} \\ 
\varepsilon _{t-1}^{\pi } \\ 
\varepsilon _{t-1}^{z} \\ 
\varepsilon _{t-1}^{y^{\ast }} \\ 
\varepsilon _{t-1}^{g}%
\end{array}%
\right] }_{X_{t-1}}+\underbrace{%
\begin{bmatrix}
0 & 0 & 0 & \sigma _{y^{\ast }} & 0 \\ 
0 & 0 & 0 & 0 & 0 \\ 
0 & 0 & 0 & 0 & \sigma _{g} \\ 
0 & 0 & \sigma _{z} & 0 & 4c\sigma _{g} \\ 
0 & 0 & 0 & 0 & 0 \\ 
1 & 0 & 0 & 0 & 0 \\ 
0 & 1 & 0 & 0 & 0 \\ 
0 & 0 & 1 & 0 & 0 \\ 
0 & 0 & 0 & 1 & 0 \\ 
0 & 0 & 0 & 0 & 1%
\end{bmatrix}%
}_{C}\underbrace{%
\begin{bmatrix}
\varepsilon _{t}^{\tilde{y}} \\ 
\varepsilon _{t}^{\pi } \\ 
\varepsilon _{t}^{z} \\ 
\varepsilon _{t}^{y^{\ast }} \\ 
\varepsilon _{t}^{g}%
\end{bmatrix}%
}_{\varepsilon _{t}}
\end{align}%
\esq

\subsubsection{Correlation between true change in natural rate and estimate}

The correlation between the true and estimated $\Delta r_{t}^{\ast }$ from
the SSM can be constructed from the relation:%
\begin{equation}
\rho =0.5\frac{\mathrm{Var}(\Delta r_{t}^{\ast })+\mathrm{Var}(E_{T}\Delta
r_{t}^{\ast })-\phi }{\sigma (\Delta r_{t}^{\ast })\sigma (E_{T}\Delta
r_{t}^{\ast })},
\end{equation}
where $\mathrm{Var}(\Delta r_{t}^{\ast })=4^{2}c^{2}\sigma _{5}^{2}+\sigma
_{3}^{2}$, $\sigma (\Delta r_{t}^{\ast })=\sqrt{\mathrm{Var}(\Delta
r_{t}^{\ast })}$, and $\mathrm{Var}(E_{T}\Delta r_{t}^{\ast })$ can be
computed from simulating from the true model, applying the Kalman Filter and
Smoother to get $E_{T}\Delta r_{t}^{\ast }$ and then computing the sample
variance of $E_{T}\Delta r_{t}^{\ast }$ as an estimate of $\mathrm{Var}%
(E_{T}\Delta r_{t}^{\ast })$.

To obtain $\phi $, add $\Delta r_{t}^{\ast }$ to the state-vector $X_{t}$
and augment the remaining matrices to be conformable. The required $\phi $
term is then the entry of $\mathrm{diag}(P_{t|T}^{\ast })$ that corresponds
to $\Delta r_{t}^{\ast }$, which will be the very last element.

The augmented SSF of \ref{K0SSM} is then: 
\begin{align}
\mathsf{Measurement}:\quad Z_{t}& =D_{1}X_{t}+D_{2}X_{t-1}+R\varepsilon _{t}
\notag \\[-4mm]
\underbrace{\left[ 
\begin{array}{c}
Z_{1t} \\ 
Z_{2t}%
\end{array}%
\right] }_{Z_{t}}& =\underbrace{%
\begin{bmatrix}
1 & 0 & 0 & 0 & 0 & \kappa _{t}\sigma _{\tilde{y}} & 0 & 0 & 0 & 0 & %
\color{red}{0} \\ 
0 & -b_{y} & 0 & 0 & 0 & 0 & \kappa _{t}\sigma _{\pi } & 0 & 0 & 0 & %
\color{red}{0}%
\end{bmatrix}%
}_{D_{1}}\underbrace{\left[ 
\begin{array}{c}
y_{t}^{\ast } \\ 
y_{t-1}^{\ast } \\ 
g_{t} \\ 
r_{t}^{\ast } \\ 
r_{t-1}^{\ast } \\ 
\varepsilon _{t}^{\tilde{y}} \\ 
\varepsilon _{t}^{\pi } \\ 
\varepsilon _{t}^{z} \\ 
\varepsilon _{t}^{y^{\ast }} \\ 
\varepsilon _{t}^{g} \\ 
\color{red}{\Delta r_t^*}%
\end{array}%
\right] }_{X_{t}} \\
& +\underbrace{%
\begin{bmatrix}
-a_{y,1} & -a_{y,2} & 0 & -\frac{a_{r}}{2} & -\frac{a_{r}}{2} & 0 & 0 & 0 & 0
& 0 & \color{red}{0} \\ 
0 & 0 & 0 & 0 & 0 & 0 & 0 & 0 & 0 & 0 & \color{red}{0}%
\end{bmatrix}%
}_{D_{2}}\underbrace{\left[ 
\begin{array}{c}
y_{t-1}^{\ast } \\ 
y_{t-2}^{\ast } \\ 
g_{t-1} \\ 
r_{t-1}^{\ast } \\ 
r_{t-2}^{\ast } \\ 
\varepsilon _{t-1}^{\tilde{y}} \\ 
\varepsilon _{t-1}^{\pi } \\ 
\varepsilon _{t-1}^{z} \\ 
\varepsilon _{t-1}^{y^{\ast }} \\ 
\varepsilon _{t-1}^{g} \\ 
\color{red}{\Delta r_{t-1}^*}%
\end{array}%
\right] }_{X_{t-1}}+\underbrace{\boldsymbol{0}_{2\times 5}}_{R}\underbrace{%
\begin{bmatrix}
\varepsilon _{t}^{\tilde{y}} \\ 
\varepsilon _{t}^{\pi } \\ 
\varepsilon _{t}^{z} \\ 
\varepsilon _{t}^{y^{\ast }} \\ 
\varepsilon _{t}^{g}%
\end{bmatrix}%
}_{\varepsilon _{t}} \\
\mathsf{State}:\quad X_{t}& =AX_{t-1}+C\varepsilon _{t},  \notag \\[5mm]
\underbrace{\left[ 
\begin{array}{c}
y_{t}^{\ast } \\ 
y_{t-1}^{\ast } \\ 
g_{t} \\ 
r_{t}^{\ast } \\ 
r_{t-1}^{\ast } \\ 
\varepsilon _{t}^{\tilde{y}} \\ 
\varepsilon _{t}^{\pi } \\ 
\varepsilon _{t}^{z} \\ 
\varepsilon _{t}^{y^{\ast }} \\ 
\varepsilon _{t}^{g} \\ 
\color{red}{\Delta r_t^*}%
\end{array}%
\right] }_{X_{t}}& =\underbrace{%
\begin{bmatrix}
1 & 0 & 1 & 0 & 0 & 0 & 0 & 0 & 0 & 0 & \color{red}{0} \\ 
1 & 0 & 0 & 0 & 0 & 0 & 0 & 0 & 0 & 0 & \color{red}{0} \\ 
0 & 0 & 1 & 0 & 0 & 0 & 0 & 0 & 0 & 0 & \color{red}{0} \\ 
0 & 0 & 0 & 1 & 0 & 0 & 0 & 0 & 0 & 0 & \color{red}{0} \\ 
0 & 0 & 0 & 1 & 0 & 0 & 0 & 0 & 0 & 0 & \color{red}{0} \\ 
0 & 0 & 0 & 0 & 0 & 0 & 0 & 0 & 0 & 0 & \color{red}{0} \\ 
0 & 0 & 0 & 0 & 0 & 0 & 0 & 0 & 0 & 0 & \color{red}{0} \\ 
0 & 0 & 0 & 0 & 0 & 0 & 0 & 0 & 0 & 0 & \color{red}{0} \\ 
0 & 0 & 0 & 0 & 0 & 0 & 0 & 0 & 0 & 0 & \color{red}{0} \\ 
0 & 0 & 0 & 0 & 0 & 0 & 0 & 0 & 0 & 0 & \color{red}{0} \\ 
\color{red}{0} & \color{red}{0} & \color{red}{0} & \color{red}{0} & %
\color{red}{0} & \color{red}{0} & \color{red}{0} & \color{red}{0} & %
\color{red}{0} & \color{red}{0} & \color{red}{0}%
\end{bmatrix}%
}_{A}\underbrace{\left[ 
\begin{array}{c}
y_{t-1}^{\ast } \\ 
y_{t-2}^{\ast } \\ 
g_{t-1} \\ 
r_{t-1}^{\ast } \\ 
r_{t-2}^{\ast } \\ 
\varepsilon _{t-1}^{\tilde{y}} \\ 
\varepsilon _{t-1}^{\pi } \\ 
\varepsilon _{t-1}^{z} \\ 
\varepsilon _{t-1}^{y^{\ast }} \\ 
\varepsilon _{t-1}^{g} \\ 
\color{red}{\Delta r_{t-1}^*}%
\end{array}%
\right] }_{X_{t-1}}+\underbrace{%
\begin{bmatrix}
0 & 0 & 0 & \sigma _{y^{\ast }} & 0 \\ 
0 & 0 & 0 & 0 & 0 \\ 
0 & 0 & 0 & 0 & \sigma _{g} \\ 
0 & 0 & \sigma _{z} & 0 & 4c\sigma _{g} \\ 
0 & 0 & 0 & 0 & 0 \\ 
1 & 0 & 0 & 0 & 0 \\ 
0 & 1 & 0 & 0 & 0 \\ 
0 & 0 & 1 & 0 & 0 \\ 
0 & 0 & 0 & 1 & 0 \\ 
0 & 0 & 0 & 0 & 1 \\ 
\color{red}{0} & \color{red}{0} & \color{red}{\sigma _{z}} & \color{red}{0}
& \color{red}{4c\sigma _{g}}%
\end{bmatrix}%
}_{C}\underbrace{%
\begin{bmatrix}
\varepsilon _{t}^{\tilde{y}} \\ 
\varepsilon _{t}^{\pi } \\ 
\varepsilon _{t}^{z} \\ 
\varepsilon _{t}^{y^{\ast }} \\ 
\varepsilon _{t}^{g}%
\end{bmatrix}%
}_{\varepsilon _{t}}
\end{align}

\end{document}
